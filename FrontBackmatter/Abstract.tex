%*******************************************************
% Abstract
%*******************************************************
%\renewcommand{\abstractname}{Abstract}
\pdfbookmark[1]{Abstract}{Abstract}
% \addcontentsline{toc}{chapter}{\tocEntry{Abstract}}
\begingroup
\let\clearpage\relax
\let\cleardoublepage\relax
\let\cleardoublepage\relax

\chapter*{Summary}
The necessity for \emph{simulations} is shared by many, if not all, fields of physics. Specifically, in \emph{High Energy Physics} this necessity is of foremost importance, due to the complexity of the experiments and the vast amount of real data which are to be compared to theoretical models. However, starting from the physical calculations, the interaction with the detectors and the reconstruction of physical objects has proven to be extremely \emph{computationally expensive and slow}. Because of this, current collaborations, such as the \emph{Compact Muon Solenoid} (CMS) one, are actually \emph{limited} by both the \emph{amount} of simulated data, the \emph{speed} at which we are capable of performing simulation and the \emph{resources} needed for such a task. As it has already been the case in countless applications, novel \emph{Machine Learning} (ML) techniques are expected to provide us with the much needed speed and accuracy, an expectation that we thoroughly investigated in the present work. The primary concern of this Thesis has been trying to build a prototype \emph{end-to-end sample analysis} generator, named \emph{FlashSim}, starting from physical process inputs and returning samples directly in the final standard analysis format at CMS (\emph{NanoAOD}), completely bypassing the actual Monte Carlo simulations. We targeted two major classes of physical objects, \emph{jets} and \emph{muons}, which would then also enable us to verify our results against the standard simulations already produced for the VBF Channel of H$\rightarrow\mu^+\mu^-$.


The results are indeed confirming and possibly exceeding our initial expectation. Through the powerful ML technique of \emph{Normalizing Flows}, we simultaneously generate 22 key target variables for muons and 17 for jets, starting from \emph{random noise} and \emph{Feynman} diagram-level physical inputs about the underlying t$\overline{\text{t}}$ process extracted directly form existing NanoAODs. The results are compared to the corresponding standard simulations results, showing optimal accuracy and preserving all the correct \emph{correlations} between single variables. 
The capacity of our approach to vary its outputs according to the specified physical content of an event is compared to the other major competing approach for fast simulation and it is found to be vastly superior. The proposed approach additionally demonstrates a raw generation speed of \emph{six orders of magnitude} greater than that of the standard approach, outputting events at a rate of 33,300 Hz against 1 event per minute. After introducing the preprocessing and postprocessing steps needed for a full end-to-end FlashSim NanoAOD generator, we apply it to a complete dataset consisting of different, previously unseen physical processes and we produce a full dataset ready to be used in the VBF H$\rightarrow\mu^+\mu^-$ analysis. We repeat the analysis performed by CMS in 2018, observing good agreement between selected-objects distributions. We obtain compatible outputs between our approach and the standard simulation from the actual \emph{Deep Neural Networks} used in the paper to perform the final signal fit, proving that the proposed approach can in fact be employed in a real-case scenario with a fraction of the time and the resources.
The current findings have the potential to completely change the approach to simulations at CMS and at the LHC, paving the way for online, on-demand generation of events. Despite this, our work also points to specific limitations, such as the current absence of \emph{fakes}, which are to be addressed for the method to see wide adoption. All these results point to interesting and rewarding directions for future research at the boundaries of high energy physics and machine learning.

This Thesis is structured into three parts:

Part 1 presents the context for our work, with Chapter 1 giving an overview of LHC, the CMS Experiment and its physics searches, focusing on the VBF Channel of H$\rightarrow\mu^+\mu^-$. Chapter 2 discusses the current approach to simulation, its costs and main limitations as well as presenting the NanoAOD format.

Part 2 explains the ML tools employed as the backbone of our work, first in a broad and general introduction in Chapter 3 and then with a focus on Normalizing Flows during Chapter 4.

Part 3 presents our contribution, discussing the implementation and the results in Chapter 5, showing the real analysis use case comparison in Chapter 6 and expanding on the conclusion and future outlook in Chapter 7. 
\endgroup

\vfill
