% **************************************************************************************************************
% A Classic Thesis Style
% An Homage to The Elements of Typographic Style
%
% Copyright (C) 2018 André Miede and Ivo Pletikosić
%
% If you like the style then I would appreciate a postcard. My address
% can be found in the file ClassicThesis.pdf. A collection of the
% postcards I received so far is available online at
% http://postcards.miede.de
%
% License:
% This program is free software; you can redistribute it and/or modify
% it under the terms of the GNU General Public License as published by
% the Free Software Foundation; either version 2 of the License, or
% (at your option) any later version.
%
% This program is distributed in the hope that it will be useful,
% but WITHOUT ANY WARRANTY; without even the implied warranty of
% MERCHANTABILITY or FITNESS FOR A PARTICULAR PURPOSE.  See the
% GNU General Public License for more details.
%
% You should have received a copy of the GNU General Public License
% along with this program; see the file COPYING.  If not, write to
% the Free Software Foundation, Inc., 59 Temple Place - Suite 330,
% Boston, MA 02111-1307, USA.
%
% PLEASE SEE ALSO THE AUTHORS' NOTE REGARDING THIS LICENSE
% IN THE DOCUMENTATION (ClassicThesis.pdf --> Chapter 1 / Chapter01.tex)
% **************************************************************************************************************
\RequirePackage{silence} % :-\
    \WarningFilter{scrreprt}{Usage of package `titlesec'}
    %\WarningFilter{scrreprt}{Activating an ugly workaround}
    \WarningFilter{titlesec}{Non standard sectioning command detected}
\documentclass[ twoside,openright,titlepage,numbers=noenddot,%1headlines,
                headinclude,footinclude,cleardoublepage=empty,abstract=on,
                BCOR=5mm,paper=a4,fontsize=11pt
                ]{scrreprt}

%********************************************************************
% Note: Make all your adjustments in here
%*******************************************************
\input{classicthesis-config}

%********************************************************************
% Bibliographies
%*******************************************************
\addbibresource{Bibliography.bib}
\addbibresource[label=ownpubs]{AMiede_Publications.bib}

%********************************************************************
% Hyphenation
%*******************************************************
%\hyphenation{put special hyphenation here}

% ********************************************************************
% GO!GO!GO! MOVE IT!
%*******************************************************
\begin{document}
\frenchspacing
\raggedbottom
\selectlanguage{american} % american ngerman
%\renewcommand*{\bibname}{new name}
%\setbibpreamble{}
\pagenumbering{roman}
\pagestyle{plain}
%********************************************************************
% Frontmatter
%*******************************************************
%\include{FrontBackmatter/DirtyTitlepage}
%*******************************************************
% Titlepage
%*******************************************************
\begin{titlepage}
    %\pdfbookmark[1]{\myTitle}{titlepage}
    % if you want the titlepage to be centered, uncomment and fine-tune the line below (KOMA classes environment)
    \begin{addmargin}[-1cm]{-3cm}
    \begin{center}
        \large

        \hfill

        \vfill

        \begingroup
            \color{CTtitle}\spacedallcaps{\myTitle} \\ \medskip
        \endgroup
        \mySubtitle \\ \bigskip

        \spacedlowsmallcaps{\myName} \\
        Supervisor: Prof. \mySupervisor

        \vfill

        \includegraphics[width=6cm]{gfx/cherubino_pant541.eps} \\ \medskip

        %\myDegree \\
        \myDepartment \\
        %\myFaculty \\
        \myUni \\ \bigskip

        \myTime\

        \vfill

    \end{center}
  \end{addmargin}
\end{titlepage}

\include{FrontBackmatter/Titleback}
\cleardoublepage%*******************************************************
% Dedication
%*******************************************************
\thispagestyle{empty}
\phantomsection
\pdfbookmark[1]{Dedication}{Dedication}

\vspace*{6cm}



Dedicato alla affettuosa memoria dei miei nonni Paolo e Giovacchino e di mia nonna Anna, \\
E a mia nonna Piera.\bigskip
    
Dedicated to the loving memory of my grandparents Paolo and Giovacchino and of my grandmother Anna,\\
And to my grandmother Piera.


%\cleardoublepage\include{FrontBackmatter/Foreword}
\cleardoublepage%*******************************************************
% Abstract
%*******************************************************
%\renewcommand{\abstractname}{Abstract}
\pdfbookmark[1]{Abstract}{Abstract}
% \addcontentsline{toc}{chapter}{\tocEntry{Abstract}}
\begingroup
\let\clearpage\relax
\let\cleardoublepage\relax
\let\cleardoublepage\relax

\chapter*{Summary}
The necessity for \emph{simulations} is shared by many, if not all, fields of physics. Specifically, in \emph{High Energy Physics} this necessity is of foremost importance, due to the complexity of the experiments and the vast amount of real data which are to be compared to theoretical models. However, starting from the physical calculations, the interaction with the detectors and the reconstruction of physical objects has proven to be extremely \emph{computationally expensive and slow}. Because of this, current collaborations, such as the \emph{Compact Muon Solenoid} (CMS) one, are actually \emph{limited} by both the \emph{amount} of simulated data, the \emph{speed} at which we are capable of performing simulation and the \emph{resources} needed for such a task. As it has already been the case in countless applications, novel \emph{Machine Learning} (ML) techniques are expected to provide us with the much needed speed and accuracy, an expectation that we thoroughly investigated in the present work. The primary concern of this Thesis has been trying to build a prototype \emph{end-to-end sample analysis} generator, named \emph{FlashSim}, starting from physical process inputs and returning samples directly in the final standard analysis format at CMS (\emph{NanoAOD}), completely bypassing the actual Monte Carlo simulations. We targeted two major classes of physical objects, \emph{jets} and \emph{muons}, which would then also enable us to verify our results against the standard simulations already produced for the VBF Channel of H$\rightarrow\mu^+\mu^-$.


The results are indeed confirming and possibly exceeding our initial expectation. Through the powerful ML technique of \emph{Normalizing Flows}, we simultaneously generate 22 key target variables for muons and 17 for jets, starting from \emph{random noise} and \emph{Feynman} diagram-level physical inputs about the underlying t$\overline{\text{t}}$ process extracted directly form existing NanoAODs. The results are compared to the corresponding standard simulations results, showing optimal accuracy and preserving all the correct \emph{correlations} between single variables. 
The capacity of our approach to vary its outputs according to the specified physical content of an event is compared to the other major competing approach for fast simulation and it is found to be vastly superior. The proposed approach additionally demonstrates a raw generation speed of \emph{six orders of magnitude} greater than that of the standard approach, outputting events at a rate of 33,300 Hz against 1 event per minute. After introducing the preprocessing and postprocessing steps needed for a full end-to-end FlashSim NanoAOD generator, we apply it to a complete dataset consisting of different, previously unseen physical processes and we produce a full dataset ready to be used in the VBF H$\rightarrow\mu^+\mu^-$ analysis. We repeat the analysis performed by CMS in 2018, observing good agreement between selected-objects distributions. We obtain compatible outputs between our approach and the standard simulation from the actual \emph{Deep Neural Networks} used in the paper to perform the final signal fit, proving that the proposed approach can in fact be employed in a real-case scenario with a fraction of the time and the resources.
The current findings have the potential to completely change the approach to simulations at CMS and at the LHC, paving the way for online, on-demand generation of events. Despite this, our work also points to specific limitations, such as the current absence of \emph{fakes}, which are to be addressed for the method to see wide adoption. All these results point to interesting and rewarding directions for future research at the boundaries of high energy physics and machine learning.

This Thesis is structured into three parts:

Part 1 presents the context for our work, with Chapter 1 giving an overview of LHC, the CMS Experiment and its physics searches, focusing on the VBF Channel of H$\rightarrow\mu^+\mu^-$. Chapter 2 discusses the current approach to simulation, its costs and main limitations as well as presenting the NanoAOD format.

Part 2 explains the ML tools employed as the backbone of our work, first in a broad and general introduction in Chapter 3 and then with a focus on Normalizing Flows during Chapter 4.

Part 3 presents our contribution, discussing the implementation and the results in Chapter 5, showing the real analysis use case comparison in Chapter 6 and expanding on the conclusion and future outlook in Chapter 7. 
\endgroup

\vfill

%\cleardoublepage\include{FrontBackmatter/Publications}
\cleardoublepage%*******************************************************
% Acknowledgments
%*******************************************************
\pdfbookmark[1]{Acknowledgments}{acknowledgments}

\begingroup
\let\clearpage\relax
\let\cleardoublepage\relax
\let\cleardoublepage\relax
\chapter*{Acknowledgments}
I am most grateful to all the people who studied with me during these years, helping me understand something new as well as providing a constant source of motivation and inspiration.
I would like to thank my family, especially my mother Maria, my father Marco and my sister Marta for always beings supportive of myself and of my choices.
My supervisor has to be thanked as well, for keeping me on track and pushing me towards results that at first I did not think possible. Many thanks to Dr. Nadya Chernyavskaya of CERN for her invaluable comments which saved me a lot of time and mistakes.

Finally, I must thank Dr. Stephen Green for his outstanding work on Normalizing Flows for likelihood-free inference and for making it publicly available. Without it, this Thesis would be something very different (and certainly worse!). My dear friend and colleague Lucia Papalini deserves an honorary mention for introducing me, albeit rather indirectly, to that piece of work.


\endgroup

\cleardoublepage\include{FrontBackmatter/Contents}
%********************************************************************
% Mainmatter
%*******************************************************
\cleardoublepage
\pagestyle{scrheadings}
\pagenumbering{arabic}
%\setcounter{page}{90}
% use \cleardoublepage here to avoid problems with pdfbookmark
\cleardoublepage
\ctparttext{This first part serves as an introduction to both the LHC and the CMS experiment, their design and their applications. We also discuss the problem of \emph{event simulation} in HEP, revising the current state-of-the-art procedure along with its main characteristics and limitations. The following two chapters are intended for putting into context the work done in this thesis.

Chapter 1 may be safely skipped if the reader has already a basic understanding of the LHC and a general-purpose apparatus such as CMS. The latter part of the chapter is the most significant, as it present the process chosen as benchmark for our simulation approach. Chapter 2 presents the specific approach to simulation and data formats taken by CMS, a necessary background for understanding the advances presented in this work.}
\part{Introduction and Problem overview}\label{pt:probover}
%************************************************
\chapter{High Energy Physics at the LHC}\label{ch:plhc} % $\mathbb{ZNR}$
%************************************************

\section{The LHC}

The Large Hadron Collider (LHC) is a circular proton-proton accelerator, located at
CERN, on the Swiss-French border. It is currently the largest and most powerful
particles accelerator ever built. The LHC is built inside the tunnel previously used
for the Large Electron-Positron Collider (LEP). The tunnel is about 27 km in length,
it is located about 100 m underground and the circle rests on a plane inclined
at 1.4$\%$ respect to gravity.

\paragraph{Design and relevant characteristics}

The LHC tunnel has not exactly a circular shape but is formed by 8 rectilinear sections and 8 circular portions. Each arc has an internal radius of 3.7 km and contains
magnetic dipoles that are used to curve the beams reaching a maximum magnetic
field of 8.33 T. The rectilinear sections in the tunnel are approximately 528 m long and contain magnetic quadrupole in order to compensate the transverse dispersion of the
protons in the bunches, caused by their mutual repulsion and by synchrotron radi-
ation.


Four major experimental insertions are located in the rectilinear section. ATLAS (A Toroidal
LHC Apparatus) and CMS (Compact Muon Solenoid) are both multi-purpose experiments, designed to
investigate a wide range of scenarios. Their focus includes Higgs boson properties
study and new physics properties searches at TeV scale. They have similar structures and similar subdetectors in order to have comparable results and to have the
possibility to cross check each other’s studies. ALICE (A Large Ion Collider Experiment) studies a phase of matter called quark-gluon plasma that is formed in
high energy nuclear collisions. Those studies are performed collecting heavy-ion
collisions. LHCb is an experiment designed to perform precision measurement of processes related to b-quark and CP-symmetry violation. Figure \ref{fig:cernacc} shows the entirety of the CERN accelerator complex

\begin{figure}
    \centering
    \includegraphics[width=\columnwidth]{gfx/ch1/CERN's-accelerator-complex2013.jpg}
    \caption[The LHC]{The LHC along with its accelerator complex.}
    \label{fig:cernacc}
\end{figure}

Protons are accelerated and injected in the LHC by a chain of four accelerators.
The first one is a linear accelerator that injects particles in a chain of three circular
accelerators. Finally, they are injected in the LHC with opposite directions in
two separate beam pipes with an energy of 450 GeV. The particles are accelerated
inside LHC by \emph{radio-frequency cavities} from 450 GeV to 7 TeV, therefore
the center of mass energy of particles collision is 14 TeV. They circulate
grouped in \emph{bunches}; each beam pipe contains about 2800 bunches in LHC and each
proton bunch has $\approx 10^{11}$ protons. The bunches are spaced 25 ns in time in the nominal
design therefore collisions happen every 25 ns, for a final event rate of about 40 MHz.

\paragraph{Luminosity and pile-up}

The collision rate of an accelerator is measured by its \emph{luminosity}. The luminosity $L$
is the ratio of the rate of produced events to the interaction cross section, $R = L \cdot \sigma$.

Searching rare events requires a large number of collisions; for this reason, the
beams are \emph{compressed} in the transverse plane just before entering in the experiments, in order to increase the luminosity of each bunch crossing. Rare events searches need high luminosity, but at high luminosity the rate
of minimum bias collisions exceeds the
bunch crossing rate and there is more
than one minimum bias interaction per
bunch crossing. This effect is known
as \emph{pile-up}. Pile-up interactions produce noise and tracks that make object
reconstruction more difficult, therefore
a compromise has to be found: the higher
the luminosity is, the rarer the events that can
be searched, but simultaneously we also have a worse event reconstruction.

\paragraph{Reference frame}

The coordinate system used by the experiments at the LHC has its origin fixed at the nominal collision point. The x axis points towards the center of the LHC ring, the y axis points
upwards and the z axis points along the counter-clockwise beam direction. The \emph{azimuthal}
angle $\phi$ is measured from the positive x direction in the xy plane and the \emph{polar} angle $\theta$ is
measured from the positive z direction. The coordinate r usually indicates the distance from
the beam line ($r = \sqrt{x^2 + y^2}$)

In a typical collision, the center-of-mass of the interaction process is boosted along the
z axis with respect to the laboratory frame. The kinematics of the collision products are
therefore conveniently described by the coordinates ($p_T$ , $y$, $\phi$, $m$). Here, $\phi$ indicates the
azimuthal angle, $m$ the invariant particle mass, $p_T$ the transverse momentum given by $p_T =
p\sin\theta = \sqrt{p_x^2 + p_y^2}$ , and $y$ the \emph{rapidity} defined as:

\begin{equation*}
    y = \frac{1}{2}\ln(\frac{E + p_z}{E - p_z})
\end{equation*}

The transverse momentum, the azimuthal angle and the mass are invariant under boosts
along the z direction, while the rapidity is simply additive. The difference in rapidity between two particles is therefore invariant under boosts along the z direction.
The rapidity can be approximated for ultra-relativistic particles by the \emph{pseudo-rapidity} $\eta$:

\begin{equation*}
    \eta = \frac{1}{2}\ln(\frac{\abs{p}+ p_z}{\abs{p} - p_z}) = -\ln(\tan\frac{\theta}{2}) 
\end{equation*}

which is computed using just the polar angle $\theta$ and is invariant under change of reference frame.

\section{The CMS Experiment}

The Compact Muon Solenoid (CMS) is a barrel shaped detector, centered at the
nominal point where the beams collide. It consists of a central part, called \emph{barrel},
and two external parts called \emph{endcaps}, placed at the ends of the cylindrical barrel.

CMS is a complex machine composed of many \emph{subdetectors}--it is illustrated in Figure \ref{fig:cms}.

\begin{figure}
    \centering
    \includegraphics[width=\columnwidth]{gfx/ch1/cms_160312_02.png}
    \caption[CMS]{The CMS experiment and its main components}
    \label{fig:cms}
\end{figure}

The heart of CMS is its superconducting solenoid that provides a magnetic field of
3.8 T with the axis aligned along the beams direction. The bore is 13 m long and
it has a radius of 3 m. The magnetic flux is returned by an iron yoke composed by
3 wheel in the barrel and three disk in each endcaps. The flux saturates the 1.5 m
of iron and ensures enough bending power to measure the momentum of the highly energetic muons that
cross the detector.

Starting from the interaction point a particle crosses the\emph{ silicon tracker}, the \emph{electromagnetic calorimeter} (ECAL) and the \emph{hadron calorimeter} (HCAL) before reaching
the solenoid. Outside the coil the muon detection system is inserted in the iron of the
yoke of the magnet. The endcaps are shaped in layers perpendicular to the beam
line and they have ECAL and HCAL sections as well as the the muon system.

\subsection{Design}
\graffito{Do we want to add figures for each subdetector?}

What follows is a brief description of the major subdetectors. As the technical details of each subdetector are extremely subtle, have already been studied and discussed in numerous articles and are not the focus of our study, we will not discuss them in the present work. The interested reader can found a detailed reference here \cite{Collaboration_2008}. Several upgrades have already been performed during the operational history of the detector, and more are to come in preparation for the \emph{High Luminosity} phase of the LHC.

\paragraph{Tracker}

The tracker constitutes the inner part of CMS and is designed to provide a precise
and efficient measurement of the charged particle \emph{tracks}, i.e. their curvature in the magnetic field, and of the primary and secondary
interaction vertices. It is immersed in an almost homogeneous magnetic field of 3.8 T pro-
vided by the CMS solenoid.

The tracker has to be light, both in the active and dead material, in order not to alter the
trajectories of charged particles and to provide the best possible resolution. It also has to
be fast enough to take data every 25 ns (40 MHz). High granularity is necessary due to the
high particle multiplicity. However, a fast and granular detector needs adequate numbers
of readout channels and an efficient cooling, which result in dead material.
To satisfy this requirements a \emph{silicon
pixel detector} is installed in the inner region, closest to the interaction point, while \emph{silicon
microstrip detectors} are used in the outer region. The total length of the tracker is 5.8 m
and its diameter 2.5 m, and the angular coverage reaches up to $\abs{\eta}$ = 2.5. 

\paragraph{Calorimeters}

The calorimeters are located outside the tracker and inside the magnetic solenoid. They are
designed to measure the energy of electromagnetic and hadronic showers and, unlike the
tracker, they are required to completely absorb the particles in the shower for optimal energy measurements. They are therefore required to be \emph{heavy}, which translates into a large
number of radiation lengths $X_0$ for the ECAL and of interaction lengths $\lambda_I$ for the HCAL.
The full tracker material has a thickness of $\approx$ 1-2 $X_0$ and less than one $\lambda_I$ for comparison.

\begin{outline}
    \1 The ECAL is a homogeneous calorimeter made
    of 61200 PbWO4 crystals mounted in the central barrel part, completed by 7324 crystals in
    each endcap. The ECAL barrel covers the central rapidity region ($\abs{\eta}$ < 1.48) and the two
    ECAL endcaps extend the coverage up to $\abs{\eta}$ = 3. A lead/silicon-strip preshower detector is
    also installed at pseudorapidities 1.6 < $\abs{\eta}$ < 2.6. The crystals are all active material: they
    induce the shower and generate scintillation light to measure the shower energy. The scintillation light is detected by silicon avalanche photodiodes in the barrel region and vacuum
    phototriodes in the endcap region. 
    
    The typical \emph{energy reslution} is:
    
    \begin{equation*}
        \frac{\sigma_E}{E} = \frac{a}{E} + \frac{b}{\sqrt{E}} + c
        \end{equation*}
        
    where a is the noise term due the electronics and pileup, independently of the energy, b is
    the stochastic term which accounts mainly for the fluctuations in the photon conversions,
    and c is a constant term related to the energy scale calibration.
    
    \1 The HCAL is used to measure the energy of hadrons, and it
    is the only detector available to measure the energy of neutral hadrons. Its design ensures
    good hermeticity to allow the measurement of the missing transverse energy and angular
    coverage in the forward region for forward jets.
    Four regions are instrumented with HCAL detectors: the barrel hadron calorimeter (HB)
    surrounds the electromagnetic calorimeter and covers the central pseudorapidity region up
    to $\abs{\eta}$ = 1.3; the two endcap hadron calorimeters (HE) cover up to $\abs{\eta}$ = 3. The Cherenkov
    calorimeter (HF) extends the coverage up to $\abs{\eta}$ = 5 in the forward region. An array of scintillators, the outer hadron calorimeter (HO), is located outside the magnet to catch the tails
    of the hadronic shower and avoid the misidentification of muons. Contrary to ECAL, the HCAL is a sampling calorimeter: the energy is measured by scintillators alternated to brass plates used as absorbers in HB and HE.
    
    The HCAL is coarser than the ECAL and the resolution is also worse:
     the combined ECAL+HCAL resolution measured in a pion test
    beam was $\sigma_E /E = 110\%/\sqrt{E} + 9\%$.
\end{outline}

\paragraph{Muon chambers}

The muon system is located outside the solenoid and covers the pseudorapidity range
$\abs{\eta}$ < 2.4. Outside of the solenoid coil, the magnetic field flux is returned through a steel
yoke. Three steel layers are present both in the barrel and in the endcaps, alternated with
four layers of muon detectors.


The muon system provides information to identify muons and to measure the momentum
and charge of high-$p_T$ muons. Additionally, two tasks are accomplished by the muon system thanks to its good time resolution: \emph{bunch crossing} identification and \emph{muon-triggering}.
Three different gaseous detector technologies are used: \emph{drift tube} (DT) chambers and \emph{cathode strip} chambers (CSC) detect muons in the regions $\abs{\eta}$ < 1.2 and 0.9 < $\abs{\eta}$ < 2.4, respectively. They are supplemented by a system of \emph{resistive plate} chambers (RPC) covering the
range $\abs{\eta}$ < 1.6. Both the DTs and the CSCs are primarily tracking detectors, but have also
good time resolution. The RPCs instead are mainly used for their good time resolution.


\paragraph{Trigger and Track reconstruction}

The 40 MHz rate of proton-proton collisions and the pileup make it impossible to process
and store all the information provided by the detector. Most of the events are not interesting
for physics analysis in any case, due the to fact that the total proton-proton cross section
is more than 6 orders of magnitude larger compared to the cross section of interesting processes. The data needs to be reduced and selected trough a trigger system, whose crucial
aspect is a fast and efficient real-time selection to record the useful events.

In CMS the data reduction happens in two steps. The Level-1 (or L1) Trigger consists of programmable electronic devices, reducing  the event rate from an input of 40 MHz to an output of about 100 kHz. The
High Level Trigger (HLT) further decreases the event rate from about 100 kHz to about 1 kHz for data storage. The HLT is implemented by a computer farm composed of more than 30000 CPUs
running software modules similar to the ones used for the offline reconstruction. 

Once the event has been selected by the trigger, other crucial processing aspects are the \emph{track reconstruction}, the \emph{primary vertex reconstruction} and the \emph{particle flow} algorithm for particle identification. For the sake of brevity, we do not discuss them here.

\paragraph{Luminosity and Pile-up}

 The CMS designed luminosity of 1034 cm$^{-1}$ s$^{–1}$ has been achieved and exceeded in the last three years of data taking. Integrated luminosity recorded by CMS and employable for physics analyses are 35.9 fb$^{–1}$ for 2016, 41.53 fb$^{–1}$ for 2017, and 60.0 fb$^{–1}$ for 2018.

As discussed this also means an increase in pile-up: Figure \ref{fig:pileup} shows its increase over the various years.

\begin{figure}
    \centering
     \includegraphics[width=\columnwidth]{gfx/ch1/pileup_allYears_run2.pdf}
    \caption[Pile-up]{The distribution of the average number of interactions per crossing (pile-up) for pp collisions in 2011 (red), 2012 (blue), 2015 (purple), 2016 (orange), 2017 (light blue), and 2018 (navy blue). The overall mean values and the minimum bias cross sections are also shown}
    \label{fig:pileup}
\end{figure}

\section{Physics program and measuraments}

\subsection{The Higgs discovery}

\subsection{The VBF Channel}

%************************************************
\chapter{Data Analysis at the LHC}\label{ch:introduction}
%************************************************
The LHC produces -at design parameters- over 600 millions collisions ($\approx 10^9$ collisions) proton-proton per second in ATLAS or CMS detectors. The amount of data collected for each event is around 1 MB (1 Megabyte). This means that we are reaching 1 PB/s (1 \emph{Peta}byte/second) of data--far too much for any detector data acquisition system to handle.

The trigger systems alleviate the problem by reducing the amount of data by a factor of 1000 or 10000; despite this, the constant need of cutting-edge resources and solutions, both in hardware and software, have been driving extremely fruitful research in the history of the LHC.

One major necessity, whose importance and scale may not be appreciated by people outside of the field, is the one for \emph{simulations}.

\section{The necessity of simulations}

First of all, why do we need simulations in the first place?
The simulation is a crucial aspect of any high energy physics experiments. In CMS it is used both
at analysis level and for testing the algorithms before deployment with data. The simulation targets relatively rare processes originating from a hard interaction between two
proton components, which are signal or background for specific analyses. Single particles
or hadronic jets can also be simulated for specific purposes.

Simulations are a necessity if we have to test a physical theory: we need to know what reality would look like if the proposed theory was true or not. Then we can check which hypothesis our data looks like, and we can either validate or disprove the theory.

Without a proper simulation, we also would not know the expected performance of the detector and the regions of interest for our process, making it impossible to correctly operate the apparatus and obtain reasonable results.

\section{Modelling an event} \label{sec:modelling}

There are three major, distinct steps in modelling a physical event at CMS. The first one is more generally related to the physical theories and calculations, while the other two are specific to the CMS detector. The whole process is often referred to as \emph{FullSim} (full simulation).

\subsection{Event generation}

The first step to any good simulation is called \emph{generation}, that is all the calculations based on Quantum Chromodynamics which describes how the quarks and gluons inside the protons scatter off one another, how they might create new particles, and how those new particles behave after they have been created. 

Proton-proton collisions are very complex and difficult to model accurately. Protons are
composed of 3 quarks, called \emph{valence} quarks, by virtual gluons and virtual quark anti-quark
pairs coming from gluon splitting. All constituents of hadrons are generically called \emph{partons}.
During high energy collisions, the protons behave as a collection of free partons and the
hard scattering can be described at the level of parton interactions. The hadronic cross
section $\sigma_{pp}$ is calculated based on the QCD \emph{factorization} theorem. The factorization theorem
states that the hadronic cross-section $\sigma_{pp}$ is a convolution of the partonic cross section  $\hat{\sigma}_{ij}$ with the parton distribution functions (PDFs) $f_i(x)$:

\[
\sigma_{pp} = \int_{x_{min}}^1 dx_1 dx_2 \sum_{i,j}f_i(x_1)f_j(x_2)\hat{\sigma}_{ij}(x_1 p_1, x_2 p_2)
\]

where function $f_i(x)$ is probability density that a parton of type i has a fraction x of the
hadron energy. The final cross section may be evaluated by the single, non-trivial contributions $\hat{\sigma}_{ij}$.

Apart from the hard interaction, the other constituents of the proton can also interact. This
usually results in a spray of softer particles, called \emph{underlying event} (UE). Any high momentum particle involved in the collision will emit additional hard QCD radiation. Radiation
from particles before the hard interaction is called \emph{initial-state-radiation} (ISR), whereas radiation off particles produced in the collision is called \emph{final-state-radiation} (FSR). Quarks and gluons can emit additional radiation via the strong interaction. All the quarks
and gluons go through the hadronization process, forming colorless hadrons. Finally, unstable particles are going to decay. A representation of all these elements is shown in Figure \ref{fig:evgen}.          

\begin{figure}
    \centering
     \includegraphics[width=.65\linewidth]{gfx/ch2/event_800px.png}
    \caption[Event generation]{ Representation of a proton-proton collision event. The red part includes the hard interaction and the decay of the products. Initial and final state radiation are in blue. A secondary
interaction can take place, in purple, before the final-state partons hadronize. The hadronization is
represented by the green blobs, and the hadron decay in dark green. Photon radiation is in yellow. From \cite{evgen}.}
    \label{fig:evgen}
\end{figure}


All the various parts involved in this step can be summarized as:

\begin{outline}
    \1  The PDFs that are phenomenological functions computed using experimental information;
    \1  the hard scattering, computed perturbatively order by order;
    \1  the parton \emph{showering}, used to simulate additional emissions in perturbative QCD;
    \1   the hadronization, describing the transition from colored particles to hadrons, treated
        using phenomenological models;
    \1   the decay of unstable particles, modeled based on experimental data.
    
\end{outline}

The first two are usually included in \emph{Matrix Elements generators}, while the last three are
included in Parton Showering programs. Both use Monte Carlo techniques; some popular multi-purpose generators include \texttt{Pythia8}, \texttt{MadGraph} and \texttt{Sherpa}.

We can thus obtain a complete description of all the (stable) particles that come out of a collision between two protons under our theory only after a complex process involving several months of modelling and calculations.

\subsection{Detector Simulation through GEANT4}

Now that we have the particles resulting from our process, we must take the detector into account. This means simulating all the interaction processes that are going to happen between particles and matter by moving them through the detector one by one and modelling the detector’s response to each one of the particles as it goes. 
Undoubtedly, the de-facto standard for such a task is the \texttt{Geant4} toolkit (see \cite{AGOSTINELLI2003250}). It is intended to simulate the passage of particles through matter and it includes a complete range of functionality including tracking, geometry, physics models and hits. The physics processes offered cover a comprehensive range, including electromagnetic, hadronic and optical processes, a large set of long-lived particles, materials and elements, over a wide energy range starting, in some cases, from 250 eV and extending in others to the TeV energy range. It has been designed and constructed to expose the physics models utilised, to handle complex geometries, and to enable its easy adaptation for optimal use in different sets of applications. The toolkit is the result of a worldwide collaboration of physicists and software engineers, created exploiting software engineering and object-oriented technology and implemented in the \texttt{C++} programming language.

\begin{figure}
    \centering
     \includegraphics[width=\columnwidth]{gfx/ch2/cms_160518_01_Scene_2.png}
    \caption[CMS model]{ Representation of the CMS detector based on the actual \texttt{Geant4} CMS Detector Description. It accurately and precisely reproduces the geometry of all CMS detector subsystems, including the geometries of the original CMS detector, phase 1 and phase 2 upgrades.  From \cite{decmod}}
    \label{fig:decmod}
\end{figure}


However, \texttt{Geant4} can only provide us with a description of the single physical processes and materials, so we need to provide the actual \emph{detector description} (see Figure \ref{fig:decmod}). 

Every piece of the detector has to be put together, with the right material assigned to each. The full detector description has millions of volumes and hundreds of different materials. There is a selected group in CMS realizing technical drawings (and photographs, as the technical drawings are not always accurate) of the detector and translating them into something \texttt{Geant4} can use. As the simulation would be far too expensive if we considered all of the actual components, at time it is necessary to define specific \emph{shortcuts} which do not impare the reliability of the simulation too much. It’s a painstaking process, still ongoing today as a continuous refinement and improvement of the current model.

Next comes the set of physics models. The toolkit has a great variety of them that you can use and they have a default quite suited to the HEP use case. Those physics models describe each process (e.g. the photoelectric effect, Compton scattering, bremsstrahlung, ionization, multiple scattering, decays, nuclear interactions, $\dots$) for each particle. Some calculations can be very complicated and costly, so we have to choose, at this point, what physics we aree interested in. \texttt{Geant4} can be used for simulation of space, simulation of cells and DNA, and simulations of radioactive environments. It is usual to take the fastest model whose results we cannot really distinguish from the most detailed models. That is, we turn off everything that we don’t really notice in our detector anyway.

The last part is to specify the \emph{saving} options. \texttt{Geant4} does not inherently know the difference between some silicon that is a part of a computer chip somewhere in the detector and the silicon that makes up the sensors in the inner detector. So we specify the parts of the detector that are the most crucial, called \emph{sensitive} detectors. There are a lot of technicalities to optimizing the storage, but in the end we want to write files with all the energy deposits that  \texttt{Geant4} has made, their time and location – and sometimes information (called \emph{truth}) about what really happened in the simulation, so later we can find out how good our reconstruction software was at correctly identifying photons and their conversions into electron-positron pairs, for example.

At the end of this costly step, we are left with a long list of energy deposits, times, and locations in our detector. 

\subsection{Digitization and Reconstruction} 

The last part of the simulation process consists in turning the energy deposits into the actual signal outputted by the detector-- a process called \emph{digitization}, once again specific for the CMS detector.

The simple idea is to change the energies into the detector outputs – usually times, voltages, and currents, for example. We have to build in all the detector effects that we care about. Some are well known such as Birks’ law, for example; others are more complicated, like the change in light collected from a scintillator tile in the calorimeter depending on whether the energy is deposited right in the middle or on the edge. We can use the digitization to model some of the very low-energy physics that we don’t want to have to simulate in detail with \texttt{Geant4} but want to approximate to an average. Those are effects like the spread and collection of charge in a silicon module or the drift of ionized gas towards a wire at low voltage.

Digitization is where some other effects are put in, like the pile-up, the extra proton-proton collisions in a single bunch crossing. Those are usually pre-simulated and add on top of the important signal events. We can add other background effects if we want to, like cosmic rays crossing the detector, or proton collisions with remnant gas particles floating around in the beampipe, or muons running down parallel to the beamline from protons that hit collimators upstream. 

Finally, we also have to pass our simulated data to the track reconstruction and particle flow algorithms to obtain a collection of physical objects as reconstructed by the detector readout. This is the \emph{reconstruction} step, resulting in events similar to the one shown in Figure \ref{fig:cmsev} . Detector hits after the simulation and reconstruction steps are called \emph{SimHits}
and \emph{RecHits} respectively.

\begin{figure}
    \centering
     \includegraphics[width=\columnwidth]{gfx/ch2/HIG-19-006_VBF_white.png}
    \caption[CMS Event]{Event in which a candidate Higgs boson produced by vector boson fusion (VBF) decays into two muons, with an invariant mass of 125.01 GeV and per-event mass uncertainty of 1.83 GeV. The forward jets from the VBF are depicted by the orange cones and the muons are drawn as long red lines. Courtesy of the CMS Collection.}
    \label{fig:cmsev}
\end{figure}


At the end of the process we have something that looks exactly like the real data – except we know exactly what it is, without any ambiguity. With that, we can build up some simulated data that includes all the different processes that we already know exist in nature, like the production of top quarks, W bosons, Z bosons, and the Higgs bosons. And we can build another set that has all of those, but also includes other alternative theories. The last part, which is really what much of the data analysis is concerned with, is trying to figure out what makes events under the new hypothesis different from the other ones we expect to see – and trying to isolate them from the others. We can look at the reconstructed energy in the event, the number of particles we find, any oddities like heavy particles decaying away from the collision point – anything that helps. And we have to know a relevant information about the simulation, so that we don’t end up using properties of the events that are very hard to simulate to separate new particles from known ones. This is usually the first part of almost all the data analyses at the LHC, while the last one is the \emph{unblinding}, where we eventually check the data that has all the features we request – passes all our requirements – and see whether it looks more like the null or alternate hypothesis. Sometimes it can be useful to use \emph{data-driven} methods to estimate the backgrounds (or tweak the estimates from our simulation), but almost every time it is common to start from the simulation itself.

\subsection{Computing Costs}

Due to the length and the complexity of the process, its tremendous computation costs should not come as a surprise. Figure \ref{fig:cpuusage} shows the public approved HL-LHC CMS projections for the CPU usage at CMS.

\begin{figure}
    \myfloatalign
    \subfloat[]
    {\includegraphics[width=.65\columnwidth]{gfx/ch2/cpu_cms2021.pdf}} \\
    \subfloat[]
    { \includegraphics[width=.65\columnwidth]{gfx/ch2/cpu_pie_cms2021.pdf}} 
    \caption[Computing estimates]{The total CPU utilization projections (a) and the approximate breakdown of CPU time into primary processing and analysis activities (b) for the CMS experiment.}\label{fig:cpuusage}
\end{figure}

The plots show estimates from 2021 made for the CMS contribution to the LHCC November review of HL-LHC computing and common software, which supersede previous results from CMS. Coherently with the CMS Phase-2 Upgrade HLT TDR, an integrated luminosity of 270 fb$^{-1}$ per year at 140 pileup events (5 KHz high level trigger rate) and a 1.2 million seconds heavy-ions run is considered during Run 4. For what concerns Run 5, an integrated luminosity of 350 fb$^{-1}$ per year at 200 pileup events (7.5 KHz high level trigger rate) is considered during Run 5. Two scenarios are considered: the first one is a baseline, which does not include any improvement due to ongoing R\&D activity, and the second one incorporates the most probable outcome of the ongoing R\&D activities. The blue curves (and points) show the annual projected needs, summed across Tier-0, Tier-1 and Tier-2 resource needs in each of these scenarios. The gray band shows the projected resource availability for an example scenario that extrapolates the 2018 CMS pledged resources using an annual increase in available resources of between 10$\%$ and 20$\%$. Results are derived from a bottoms-up model of CMS offline processing activities, including prompt reconstruction, Monte Carlo simulation, data re-reconstruction and all phases of analysis activities. 

We also show the approximate breakdown of CPU time into primary processing and analysis activities during a typical HL-LHC year. The plot corresponds to a snapshot of the year 2029. The baseline scenario is considered, i.e. projected effects from on-going R\&D that will reduce the computing resources needed by CMS are not considered.

The presented figures make it clear that:

\begin{outline}
    \1 Both the future and the present CPU (as well as Disk) utilization are completely dominated by the event simulation tasks, which leaves out only about 6$\%$ of resources for other necessities;
    \1 Without improvements from the R\&D sector we risk not being able to meet the future necessities of the collaboration.
\end{outline}


\subsection{Fast simulation}

Clearly, the possibility of performing fast simulations would be a major improvement over the current simulation pipeline, and there are already many different studies and approaches for doing so.

\paragraph{Top-down approach}

In this type of approach, we search for the slowest part of our process and find ways of making it as fast as possible with specific parametric shortcuts. The CMS Collaboration has actually a dedicated approach called CMS \emph{FastSim}, see \cite{https://doi.org/10.48550/arxiv.1701.03850}. 

While FullSim uses the exact detector geometry, tracks particles in small steps, and uses
detailed models for material interactions, FastSim uses a simplified geometry with infinitely thin material layers and simple analytical material interaction models that are parametrized
and tuned to agree with FullSim. For the digitization step, both FullSim and FastSim do a detailed
emulation of detector electronics and trigger, with small exceptions in FastSim. In the reconstruction step, FullSim employs the standard event reconstruction used for reconstructing the real CMS
data. FastSim uses standard reconstruction for calorimetry and muon systems, but a simplified
reconstruction for tracking, based on smearing and truth information, in order to reduce CPU time.

Performance of FastSim is validated regularly within the official CMS software release val-
idation framework; overall, distributions in FastSim agree with FullSim within $\approx 10\%$.

\paragraph{Bottom-up approach}

The bottom-up approach means that we try to skip detector simulation and digitization all together and go directly to the final objects that we would have reconstructed (electrons, muons, jets, missing transverse momentum, $\dots$). A popular framework is the one of \texttt{DELPHES} fast-simulation, see \cite{de_Favereau_2014}.

It takes as input the most common event generator output and performs a fast and realistic simulation of
a general purpose collider detector. To do so, long-lived particles emerging from the hard
scattering are propagated to the calorimeters within a uniform magnetic field parallel to
the beam direction. The particle energies are computed by smearing the initial long-lived
visible particles momenta according to the detector resolution. As a result, jets, missing
energy, isolated electrons, muons and photons, and taus can be reconstructed. It is oviously not meant to
be used for advanced detector studies, for which more accurate tools are needed, but it can nonetheless perform quickly realistic physics studies without in-depth knowledge of the technicalities.


\begin{table}
\begin{adjustbox}{width={\textwidth},totalheight={\textheight},keepaspectratio}%
    \begin{tabular}{lll} \toprule
        \tableheadline{Simulation framework} & \tableheadline{key aspects} & \tableheadline{speed} \\ \midrule
        FullSim & \tabitem detailed geometry; &  $\mathcal{O}(40$ s) for a t$\overline{\text{t}}$ event \\
        & \tabitem small-step tracking; & \\
        & \tabitem \texttt{Geant4} material modelling; & \\
        & \tabitem  detailed digitization; & \\
        & \tabitem full reconstruction & \\
        \midrule
        FastSim & \tabitem simplified geometry; &  $\mathcal{O}(5$ s) for a t$\overline{\text{t}}$ event \\
        & \tabitem thin material layers; & \\
        & \tabitem simple analytical material modelling; & \\
        & \tabitem  detailed digitization; & \\
        & \tabitem full reconstruction with exceptions & \\
        %postulant quo & westeuropee & sanctificatec \\
        \midrule
        Delphes & \tabitem simple tracking system; &  $\mathcal{O}(0.01$ s) for a t$\overline{\text{t}}$ event \\
        & \tabitem smearing & \\
        %autem vulputate ex & parola & romanic \\
        %usu mucius iisque & studio & sanctificatef \\
        \bottomrule
    \end{tabular}
    \end{adjustbox}
    \caption[Simulation frameworks]{An at-glance comparison between the various simulation frameworks employed at the CMS experiment.}
    \label{tab:simfram}
\end{table}

As Table \ref{tab:simfram} shows, the various fast simulation approaches allow us to obtain significant speedups. However, while CMS FastSim accuracy is constantly being improved, its speed is still somewhat lacking; while \texttt{DELPHES} achieves a marked speedup but sacrifices the realistic detector simulation.

\section{The NanoAOD format}

As a direct consequence of how the event modelling is structured, the resulting output contains a great deal of information about the event. Aside from final state objects, we are able of keeping track of various steps as the particles pass through the detector, as well as individual components of jets, with great precision. Starting from raw data produced from the online system or the MC simulation, successive degrees of processing (the event reconstruction) refine this data, apply calibrations and create higher-level physics objects. CMS uses a number of data formats with varying degrees of detail, size, and refinement to write this data in its various stages. In turn, the data formats get grouped within an Event file into multiple Event formats, according to the data origin or content.  All of this comes at the cost of \emph{size}.

Event information from each step in the simulation and reconstruction chain is logically grouped into what we call a \emph{data tier}. Examples of data tiers include RAW and RECO, and for simulated data we also have the previously mentioned steps: GEN, SIM and DIGI. A given dataset may consist of multiple data tiers, e.g., the term GenSimDigi includes the generation (MC), the simulation (Geant) and digitization steps. The most important tiers from a physicist's point of view are probably RECO (all reconstructed objects and hits) and AOD (Analysis Object Data, a smaller subset of RECO). Figure \ref{fig:datatier} gives an overview. 

\begin{figure}
    \centering
     \includegraphics[width=\columnwidth]{gfx/ch2/whats_in_aod_reco.png}
    \caption[Data Tiers]{RAW data is the Detector data after online formatting, the L1 trigger result, the result of the HLT selections (HLT trigger bits), potentially some of the higher level quantities calculated during HLT processing. RECO data contains objects from all stages of reconstruction. AOD are derived from the RECO information to provide data for physics analyses in a convenient, compact format. Most physics analyses can run on AOD data. Courtesy of the CMS public wiki.}
    \label{fig:datatier}
\end{figure}

The AOD format, containing all the relevant physical objects, forms the backbone of the CMS analyses. However, such an amount of information mean that the typical AOD will store about 400 kB \emph{per event}. This is often redundant for the typical analysis. Additionally, the increase in luminosity during subsequent runs means that the size of the format will keep increasing: at the end of 2013, the total size of AOD stored by CMS for Run 1 was about 20 petabytes. The CMS collaboration has thus commissioned various new data formats aimed at reducing the size of the analysis data files.
Initially the MiniAOD format was introduced in 2015 (see \cite{Petrucciani_2015}), about one tenth the size of AOD (40-60 kB per event), and which has seen a wide adoption by the experiment during Run 2. In recent years, however, the \emph{NanoAOD} format has been proposed \cite{Peruzzi_2020}, with the aim of serving the needs of a substantial fraction of its physics analyses with a per-event payload of about 1-2 kB.

NanoAOD achieves such a strong data reduction by retaining only high level information on physics objects, such as jets and leptons, dropping their individual constituents, and reducing the precision of stored variables.
The design rationale of NanoAOD is based on previous experience from user ntuples and considerations on which minimal set of object information can support a large set of analysis efforts. The introduction of the NanoAOD data format has a dramatic impact on the estimates for computing resources needed by the CMS experiment during the High-Luminosity operation phase of the LHC. Assuming a design target of 50$\%$ analysis coverage with NanoAOD is met by then, the projected needs for disk storage in 2027 are decreased by a factor of about 2, corresponding to a reduction of more than 2 EB (Exabytes). The needed CPU processing power is also decreased by about 15$\%$, as user ntuple production is partially replaced by the central NanoAOD workflow.

Despite this, the potential advantages of the NanoAOD are still partially hampered by the current production procedure: NanoAOD can be produced from MiniAOD files at a rate of about 10 events per second on a
single CPU core. This, in turn, means that we are bound to have an underling AOD file, which implies that we still need to perform all the computationally expensive steps of Section \ref{sec:modelling}.
However, the consistently reduced size of the NanoAOD opens up new possibilities for its production thorugh non-conventional, non-\texttt{Geant4} based approaches. 

Specifically, in the present work we are interested to investigate \emph{state-of-the-art deep learning} techniques as a way of bypassing the Simulation and Digitization steps and directly produce new, original events in the NanoAOD format in a fraction of the time and using only a subset of the resources.

%*****************************************
%*****************************************
%*****************************************
%*****************************************
%*****************************************

\cleardoublepage
\ctparttext{We move on to describe the building blocks of our work, i.e. \emph{Deep Learning} and specifically \emph{Generative Models}. We conclude with a discussion about the powerful approach know as \emph{Normalizing Flows}. 

Chapter 3 serves as an high level introduction to the field, and may be safely skipped if the reader is already acquainted with it. Chapter 4 discusses in greater detail a specific and less well-known ML algorithm for data generation which allowed us to derive the results of the latter part of this work.}
\part{Tools}\label{pt:tools}
%\addtocontents{toc}{\protect\clearpage} % <--- just debug stuff, ignore
%************************************************
\chapter{Normalizing Flows}\label{ch:mathtest} % $\mathbb{ZNR}$
%************************************************
\begin{flushright}{\slshape
    Know from the rivers \\
    in clefts and in crevices: \\
    those in small channels flow noisily, \\
    the great flow silent. \\
    Whatever’s not full makes noise.\\
    Whatever is full is quiet. } \\ \medskip
    --- Buddha, Nalaka Sutta
\end{flushright}

In standard probabilistic modeling practice, we represent our beliefs over unknown continuous quantities with simple parametric distributions like the normal, exponential, and Laplacian distributions. However, using such simple forms, which are commonly symmetric and unimodal (or have a fixed number of modes when we take a mixture of them), restricts the performance and flexibility of our methods. For instance, standard variational inference in the Variational Autoencoder uses independent univariate normal distributions to represent the variational family. The true posterior is neither independent nor normally distributed, which results in suboptimal inference and simplifies the model that is learnt. In other scenarios, we are likewise restricted by not being able to model multimodal distributions and heavy or light tails, widespread in HEP.

\emph{Normalizing Flows} are a type of \emph{latent generative model}, capable of producing new, original samples from a latent space, usually a Gaussian one. The main advantage of this approach, when compared to the previous ones, is that it has been specifically engineered to define explicit \emph{densities}--making it particularly well suited to our use case.

The present chapter serves as a general explanation of the basic concepts and ideas for defining and implementing Normalizing Flows (NF for short). We also briefly discuss in a more general way the possible use cases of this architecture. For the interested reader, an excellent review is the one by Papamakarios et al \cite{papanf}.

\section{Definitions}

Normalizing flows are a family of methods for constructing flexible learnable probability distributions, often with neural networks, which allow us to surpass the limitations of simple parametric forms to represent complex high-dimensional distributions.

\subsection{Basics}

The basic idea is to define a complex distribution $p_x(\mathbf{x})$ by passing \emph{random variables} $\mathbf{z} \in \mathbb{R}^D$ drawn from a simple \emph{base distribution} $p_z(\mathbf{z})$ through a non-linear, \emph{invertible} transformation \emph{f}: $\mathbb{R}^D \rightarrow \mathbb{R}^D$, $\mathbf{x} = f(\mathbf{z})$. The base distribution is usually chosen to be simple, for example a standard i.i.d. normal distribution, $\mathbf{z}\sim\mathcal{N}(\mathbf{0},I_{D\times D})$, which makes it very simple to sample and evaluate. 

We may now use the change-of-variable formula to express $p_x(\mathbf{x})$ as:

\[
p_x(\mathbf{x}) = p_z(\mathbf{z})\det\left|\frac{d\mathbf{z}}{d\mathbf{x}}\right|
\]

and remembering that \emph{f} is invertible, taking the logarithm of both sides we get:

\[
	\begin{aligned}
		\log(p_x(x)) &= \log(p_z(f^{-1}(\mathbf{x})))+\log\left(\det\left|\frac{d\mathbf{z}}{d\mathbf{x}}\right|\right)\\
		&= \log(p_z(f^{-1}(\mathbf{x})))-\log\left(\det\left|\frac{d\mathbf{x}}{d\mathbf{z}}\right|\right)
	\end{aligned}
\]

where $d\mathbf{z}/d\mathbf{x}$ denotes the Jacobian matrix of $f^{-1}(\mathbf{x})$.
Intuitively, this equation says that the density of $x$ is equal to the density at the corresponding point in $z$ plus a term that corrects for the warp in volume around an infinitesimally small volume around $x$ caused by the transformation.
	We can compose such bijective transformations to produce even more complex distributions. It is clear that, if we have $L$ transforms $f_{(0)}, f_{(1)},\ldots,f_{(L-1)}$, then the log-density of the transformed variable $\mathbf{x}=(f_{(0)}\circ f_{(1)}\circ\cdots\circ f_{(L-1)})(\mathbf{z})$ is:
	
	\begin{equation*}
		\begin{aligned}
			\log(p_x(\mathbf{x})) &= \log\left(p_z\left(\left(f_{(L-1)}^{-1}\circ\cdots\circ f_{(0)}^{-1}\right)\left(\mathbf{x}\right)\right)\right)+\\
			&+\sum^{L-1}_{l=0}\log\left(\left|\frac{df^{-1}_{(l)}(\mathbf{x}_{(l)})}{d\mathbf{x}'}\right|\right)
		\end{aligned}
	\end{equation*}
	
Based on this relationship, the idea is to define some kind of divergence measure between the two pdfs, which can then be used as the objective function to minimize to learn the optimal transformation \emph{f}.

\subsection{Loss functions}

As the idea is to leverage deep learning, we let our transformation \emph{f} depend on a set of parameters $\pmb{\phi}$, $f = f(\mathbf{x}; \pmb{\phi})$.
We can now distinguish two main cases.

\paragraph{Forward KL Divergence}
Suppose that we have samples from the target distribution (or we are able to generate them), but we cannot evaluate the underlying pdf $p_x(\mathbf{x})$. This is precisely our case in HEP, with billions of available Montecarlo data and no analytical pdf.

\section{Constructing flows}

\subsection{Coupling Layers}

\subsection{Splines}

\section{Applications}

%*****************************************
%*****************************************
%*****************************************
%*****************************************
%*****************************************

%************************************************
\chapter{Normalizing Flows}\label{ch:mathtest} % $\mathbb{ZNR}$
%************************************************

In standard probabilistic modeling practice, we represent our beliefs over unknown continuous quantities with simple parametric distributions like the normal, exponential, and Laplacian distributions. However, using such simple forms, which are commonly symmetric and unimodal (or have a fixed number of modes when we take a mixture of them), restricts the performance and flexibility of our methods. For instance, standard variational inference in the Variational Autoencoder uses independent univariate normal distributions to represent the variational family. The true posterior is neither independent nor normally distributed, which results in suboptimal inference and simplifies the model that is learnt. In other scenarios, we are likewise restricted by not being able to model multimodal distributions and heavy or light tails, widespread in HEP.

\emph{Normalizing Flows} are a type of \emph{latent generative model}, capable of producing new, original samples from a latent space, usually a Gaussian one. The main advantage of this approach, when compared to the previous ones, is that it has been specifically engineered to define explicit \emph{densities}--making it particularly well suited to our use case.

The present chapter serves as a general explanation of the basic concepts and ideas for defining and implementing Normalizing Flows (NF for short). We also briefly discuss in a more general way the possible use cases of this architecture. For the interested reader, an excellent review is the one by Papamakarios et al \cite{papanf}.

\section{Definitions}

Normalizing flows are a family of methods for constructing flexible learnable probability distributions, often with neural networks, which allow us to surpass the limitations of simple parametric forms to represent complex high-dimensional distributions.

\subsection{Basics}

The basic idea is to define a complex distribution $p_x(\mathbf{x})$ by passing \emph{random variables} $\mathbf{z} \in \mathbb{R}^D$ drawn from a simple \emph{base distribution} $p_z(\mathbf{z})$ through a non-linear, \emph{invertible} and \emph{differentiable} transformation \emph{f}: $\mathbb{R}^D \rightarrow \mathbb{R}^D$, $\mathbf{x} = f(\mathbf{z})$. \emph{f} can also be called a \emph{bijection}. The base distribution is usually chosen to be simple, for example a standard i.i.d. normal distribution, $\mathbf{z}\sim\mathcal{N}(\mathbf{0},I_{D\times D})$, which makes it very simple to sample and evaluate. 

We may now use the change-of-variable formula to express $p_x(\mathbf{x})$ as:

\[
p_x(\mathbf{x}) = p_z(\mathbf{z})\det\left|\frac{d\mathbf{z}}{d\mathbf{x}}\right|
\]

and remembering that \emph{f} is invertible, taking the logarithm of both sides we get:

\begin{equation}\label{eqn:logpdf}
	\begin{aligned}
		\log(p_x(x)) &= \log(p_z(f^{-1}(\mathbf{x})))+\log\left(\det\left|\frac{d\mathbf{z}}{d\mathbf{x}}\right|\right)\\
		&= \log(p_z(f^{-1}(\mathbf{x})))-\log\left(\det\left|\frac{d\mathbf{x}}{d\mathbf{z}}\right|\right)
	\end{aligned}
\end{equation}

where $d\mathbf{z}/d\mathbf{x}$ denotes the Jacobian matrix of $f^{-1}(\mathbf{x})$, $\mathbb{J}_{f^{-1}}(\mathbf{x})$.
Intuitively, this equation says that the density of $x$ is equal to the density at the corresponding point in $z$ plus a term that corrects for the warp in volume around an infinitesimally small volume around $x$ caused by the transformation.
	We can compose such bijective transformations to produce even more complex distributions. It is clear that, if we have $K$ transforms $f_{(1)}, f_{(2)},\ldots,f_{(K)}$, then the log-density of the transformed variable $\mathbf{x}=(f_{(1)}\circ f_{(2)}\circ\cdots\circ f_{(K)})(\mathbf{z})$ is:
	
	\begin{equation*}
		\begin{aligned}
			\log(p_x(\mathbf{x})) &= \log\left(p_z\left(\left(f_{(K)}^{-1}\circ\cdots\circ f_{(1)}^{-1}\right)\left(\mathbf{x}\right)\right)\right)+\\
			&+\sum^{K}_{k=1}\log\left(\left|\frac{df^{-1}_{(k)}(\mathbf{x}_{(k)})}{d\mathbf{x}'}\right|\right)
		\end{aligned}
	\end{equation*}
	
This relationship between the base distribution and the transformed one through this chain of invertible transforms is at the core of the NF approach and is illustrated in Figure \ref{fig:nf}.

\begin{figure}
    \centering
    \includegraphics[width=\columnwidth]{gfx/ch4/normalizing-flow.png}
    \caption[Normalizing Flows]{The main idea behind Normalizing Flows: how can we find a chain of invertible transformations to send $p_z(\mathbf{z})$ to $p_x(\mathbf{x})$? Taken from \cite{nffig}.}
    \label{fig:nf}
\end{figure}
	
Based on this, the idea is to define some kind of measure, which can then be used as the objective function to minimize, to learn the optimal transformation \emph{f}.

\subsection{Loss functions}

As the idea is to leverage deep learning, we let our transformation \emph{f} depend on a set of parameters $\phi$, $f = f(\mathbf{x};\, \phi)$.
For the sake of completeness, we distinguish two main cases.

\paragraph{Forward KL Divergence}
Suppose that we have samples from the target distribution (or we are able to generate them), but we cannot evaluate the underlying pdf $p_x^*(\mathbf{x})$. This is precisely our case in HEP, with billions of available Monte Carlo data and no analytical pdf. Then, we may define as our loss function the \emph{forward Kullback-Leibler divergence} between the target distribution $p_x^*(\mathbf{x})$ and the flow-defined one $p_x(\mathbf{x}; \, \phi)$:

\[
\begin{aligned}
    \mathcal{L}(\phi) &= \mathcal{D}_{KL}[p_x^*(\mathbf{x})||p_x(\mathbf{x}; \, \phi)]\\
    &= -\mathbb{E}_{p_x^*(\mathbf{x})}[\log(p_x(\mathbf{x}; \, \phi))] +\; \text{const.}\\
    &= -\mathbb{E}_{p_x^*(\mathbf{x})}[\log(p_z(f^{-1}(\mathbf{x}; \, \phi)))+\log\left(\det\mathbb{J}_{f^{-1}}(\mathbf{x}; \, \phi)\right)] +\; \text{const.}
\end{aligned}
\]

where we have used Eq. \ref{eqn:logpdf}. Supposing we had a set of training samples $\{\mathbf{x}\}^N_n$ from the target pdf, we may estimate the expectation value over $p_x^*(\mathbf{x})$ by Monte Carlo as:

\[
\mathcal{L}(\phi) \approx -\frac{1}{N} \sum_n \log(p_z(f^{-1}(\mathbf{x}; \, \phi)))+\log\left(\det\mathbb{J}_{f^{-1}}(\mathbf{x}_n; \, \phi)\right) +\; \text{const.}
\]

For computing this loss we need to calculate $f^{-1}$, its Jacobian determinant and the density $p_z(f^{-1}(\mathbf{x}; \, \phi))$. 

\paragraph{Reverse KL Divergence} If instead we have the ability to easily evaluate $p_x^*(\mathbf{x})$ but not to sample from it, the \emph{reverse Kullback-Leibler divergence} is more well suited:

\[
\begin{aligned}
    \mathcal{L}(\phi) &= \mathbb{E}_{p_x(\mathbf{x}, \, \phi)}[\log(p_x(\mathbf{x}; \, \phi)) - \log(p_x^*(\mathbf{x}))]\\
    &= \mathbb{E}_{p_x(\mathbf{x}, \, \phi)}[\log(p_z(\mathbf{z}))-\log\left(\det\mathbb{J}_{f}(\mathbf{x}; \, \phi)\right)-\log(p_x(\mathbf{x}; \, \phi))]
\end{aligned}
\]

\section{Constructing flows}
For an actual implementation, the main challenge is in designing parametrizable multivariate bijections that have closed form expressions for both $f$ and $f^{-1}$, a tractable Jacobian whose calculation scales with $O(D)$ rather than $O(D^3)$, and can express a flexible class of functions.

\subsection{Coupling Layers}

One simple way to reduce the computational complexity of the Jacobian is to introduce \emph{coupling layers}. The basic idea is illustrated in Figure \ref{fig:coupla}.

\begin{figure}
    \centering
    \scalebox{0.65}{
    \begin{tikzpicture}[
        thick, node distance=15mm,
        set/.style={draw, diamond, text width=8mm, align=center},
        op/.style={draw, circle, text width=5mm, align=center, fill=orange!40},
      ]
    
      \node[set, fill=blue!20] (z1) {$\vec z_{1:d}$};
      \node[op, right=of z1] (eq) {\raisebox{-1ex}=};
      \node[set, right=of eq, fill=blue!20] (x1) {$\vec x_{1:d}$};
      \draw[->] (z1) edge (eq) (eq) edge (x1);
    
      \node[set, below=1 of z1, fill=green!30] (z2) {$\mathclap{\vec z_{d+1:D}}$};
      \node[op, right=of z2] (g) {$f$};
      \node[below=1em of g] (forward) {forward pass};
      \node[set, right=of g, fill=yellow!40] (x2) {$\mathclap{\vec x_{d+1:D}}$};
      \draw[->] (z2) edge (g) (g) edge (x2);
    
      \node[op] (m) at ($(z1)!0.5!(g)$) {$\phi$};
      \draw[->] (z1) edge (m) (m) edge (g);
    
      \begin{scope}[xshift=9cm]
    
        \node[set, fill=blue!20] (z1) {$\vec z_{1:d}$};
        \node[op, right=of z1] (eq) {\raisebox{-1ex}=};
        \node[set, right=of eq, fill=blue!20] (x1) {$\vec x_{1:d}$};
        \draw[<-] (z1) edge (eq) (eq) edge (x1);
    
        \node[set, below=1 of z1, fill=green!30] (z2) {$\mathclap{\vec z_{d+1:D}}$};
        \node[op, right=of z2] (g) {$\mathclap{f^{-1}}$};
        \node[below=1em of g] (inverse) {inverse pass};
        \node[set, right=of g, fill=yellow!40] (x2) {$\mathclap{\vec x_{d+1:D}}$};
        \draw[<-] (z2) edge (g) (g) edge (x2);
    
        \node[op] (m) at ($(x1)!0.5!(g)$) {$\phi$};
        \draw[->] (x1) edge (m) (m) edge (g);
    
      \end{scope}
    
    \end{tikzpicture}}
    \caption[Coupling layer]{The two possible passes through a coupling layer.}
    \label{fig:coupla}
\end{figure}

We partition the input $\mathbf{z} \in \mathbb{R}^D$ into two subsets $(\vec z_{1:d}, \, \vec z_{d+1:D}) \in \mathbb{R}^d \times \mathbb{R}^{D-d}$, where \emph{d} is an integer between 1 and \emph{D}, and is usually taken as $\lceil D/2 \rceil$. Then, at each step, the transformation $f_i$ is defined as:

\[
f_i(\mathbf{z}; \, \phi) = f_i(\vec z_{d+1:D}; \, \phi(\vec z_{1:d}))
\]

that is, the single transformation acts only on a \emph{subset} of the input, keeping the other part unchaged and using it as \emph{conditioning} for the actual parameters. 

At the end, the two subset are joined together to form the input for the next layer; by applying arbitrary permutations between the indexes we can ensure that eventually all values of the input get transformed. What is more, this also means that because we condition on the different indexes of the same input the transformation can correctly learn to model \emph{correlations} between 1d distribution for the same instance.

But the greatest advantage is that now the single transformation depends only on one subset of inputs, and thus its jacoban is \emph{block triangular}:

\[
\mathbb{J}_{f}(\mathbf{x}; \, \phi) = 
\begin{pmatrix}
\frac{\partial \vec x_{1:d}}{\partial \vec z_{1:d}} & \frac{\partial \vec x_{1:d}}{\partial \vec z_{d+1:D}}\\
\frac{\partial \vec x_{d+1:D}}{\partial \vec z_{1:d}} & \frac{\partial \vec x_{d+1:D}}{\partial \vec z_{d+1:D}}
\end{pmatrix}
=
\begin{pmatrix}
\mathbb{I} & 0\\
A & \mathbb{J}^*
\end{pmatrix}
\]

The full Jacobian determinant can simply be calculated from the product of the diagonal elements of $\mathbb{J}^*$, which are simply the partial derivatives over ($d+1, \dots D$). A similar reasoning holds for the Jacobian of the inverse.
This significantly speeds up the calculations.

\subsection{Splines}

There are many possible bijections which one can use in building a NF. Recent advancements have demonstrated the suitability of \emph{rational-quadratic spline transforms} (see \cite{durkan}). 

A monotonic spline is a piecewise function consisting of K segments, where each segment is a simple function that is easy to invert. Specifically, given a set of K+1 input locations $l_{0}, \dots, l_K$ , the transformation
is taken to be a simple monotonic function (e.g. a low-degree polynomial) in each interval
[$l_{k}, l_{k+1}$], under the constraint that the K segments meet at the endpoints.
Outside the interval [$l_{0}, l_K$], the transformer can default to a simple function such as the
identity. Typically, the parameters $\phi$ of the transformations are the input locations, 
the corresponding output locations and (depending on the type of spline) the
derivatives (i.e. slopes) at $l_{0}, \dots, l_K$. An example is illustrated in Figure \ref{fig:rqs}.

\begin{figure}
    \centering
    \includegraphics[width=\columnwidth]{gfx/ch4/D9F0PDyWsAAWKHf.png}
    \caption[Rational quadratic spline]{A rational quadratic spline $g_{\theta}(x)$ and its derivative $g_{\theta}^{'}(x)$}
    \label{fig:rqs}
\end{figure}

Spline-based transformers are as fast to invert as to evaluate, while
maintaining exact analytical invertibility. Evaluating or inverting a spline-based transformer
is done by first locating the right segment--which can be done in $\mathcal{O}$(log K) time using binary
search—and then evaluating or inverting that segment, which is assumed to be analytically
tractable. By increasing the number of segments K, a spline-based transformer can be
made arbitrarily flexible.

\subsection{Conditional distributions}

The theory of Normalizing Flows is also easily generalized to conditional distributions. We denote the variable to condition on by $C=\mathbf{c}\in\mathbb{R}^M$. A simple multivariate source of noise, for example a standard i.i.d. normal distribution, $\mathbf{z}\sim\mathcal{N}(\mathbf{0},I_{D\times D})$, is passed through a vector-valued bijection that also conditions on C, $f:\mathbb{R}^D\times\mathbb{R}^M\rightarrow\mathbb{R}^D$, to produce the more complex transformed variable $\mathbf{x}=f(\mathbf{z};\, \phi(C))$. 

In practice, this is usually accomplished by making the parameters for a known normalizing flow bijection $f$ the output of a neural network that inputs $\mathbf{c}$ as well as one of the subsets of the coupling layer. It is thus straightforward to condition event generation on some ground truth, e.g. the Monte Carlo Gen values matched to our targets.


\section{Applications}

Besides the generation of samples, the use cases of Normalzing Flows are numerous. 
Here we simply limit ourselves to those that we deem more interesting from the point of view a physicist.

\paragraph{Density estimation and generation}

\paragraph{Inference}

\paragraph{Anomaly detection}
%*****************************************
%*****************************************
%*****************************************
%*****************************************
%*****************************************

\cleardoublepage
\ctparttext{This final part presents the original contribution resulting fro  this Thesis work. In chapter 5 we discuss the practical implementation and setup and show that our approach to simulation is capable of producing high-quality data samples with \emph{several orders of magnitude of speedup} when compared to FullSim.

Chapter 6 shows that we can actually employ our generated data for performing a MC-based analysis, specifically the VBF Channel of H$\rightarrow\mu^+\mu^-$.

Finally, Chapter 7 draws a conclusion of the present work, discussing some possible future research directions.}
\part{Thesis contribution}\label{pt:thcont}
%************************************************
\chapter{Flash simulation of samples with Normalizing Flows}\label{ch:fs} % $\mathbb{ZNR}$
%************************************************

Having discussed the importance and the challenges of event simulation at LHC, and having described the Deep Learning Normalizing Flows approach, we dedicate this chapter to the practical implementation of an end-to-end sample generator.

The code discussed in the following sections and used in this work may be found online \href{tbd}{here}\footnote{tbd}.

\section{Target variables}

As we discussed in Section \ref{sec:targets}, we choose to target the VBF Channel of H$\rightarrow\mu^+\mu^-$. Thanks to the clear signature, we only needed to simulate jets and muons out of all the possible objects in a NanoAOD. The present section serves as a discussion of the chosen variables to be simulated with our approach.

However, because our Machine Learning Models need a large amount of training data, having already been simulated through FullSim in large numbers, the VBF signal MC data samples were not sufficient for our task. We thus turned to the t$\overline{\text{t}}$ process.

\subsection{The t$\overline{\text{t}}$ process}
The top quark is an important component of the standard model, especially because of
its large mass, and its properties are critical for the overall understanding of the theory. Measurements of the top quark-antiquark pair (t$\overline{\text{t}}$) production cross section test the predictions of
quantum chromodynamics (QCD), constrain QCD parameters, and are sensitive to physics beyond the SM. With a cross-section of $\approx 900$ pb at 13 TeV , the t$\overline{\text{t}}$ process is also \emph{the dominant SM background to many searches for new
physical phenomena}, and its precise measurement is essential for claiming new discoveries.
The copious top quark data samples produced at the CERN LHC enable measurements of the t$\overline{\text{t}}$
production rate in extended parts of the phase space, and differentially as a function of the kinematic properties of the t$\overline{\text{t}}$ system. Inclusive and differential cross section measurements from
proton-proton (pp) collisions at centre-of-mass energies of 13 TeV have been reported by
the CMS collaboration in \cite{Sirunyan_2017}.

\graffito{were our samples dijets or mixed?}
Top quarks decay almost exclusively into a W boson and a b quark. The W may then decay in either a q$\overline{\text{q}}$ or a lepton and its corresponding neutrino, ensuring that the events will be well populated with both jets and muons, our simulation targets. \graffito{do we want a figure for ttbar?}

%\begin{figure}
    %\centering
    %\includegraphics[scale=0.2]{gfx/ch5/CCMarApr_LHC10_fig2.jpg}\quad
    %\includegraphics[scale=0.3]{gfx/ch5/feynman_ttbar_ljets_longt.png}
    %\caption[t$\overline{\text{t}}$ diagram]{ The t$\overline{\text{t}}$ process is dominating the cross sections at LHC, making it one of the leading SM background processes. Bottom: the production diagram for proce}
    %\label{fig:ttfig}
%\end{figure}

Having at our disposal a large set of FullSim MC NanoAOD samples for t$\overline{\text{t}}$ dijet/at least one jet(??) events, we used these to \emph{train} our models on the two target objects.

\subsection{Jets}

As discussed in Section \ref{sec:nanoaod}, a NanoAOD contains both the Jet objects, i.e. final-state reconstructed jets, and the GenJet objects, the jets resulting from a generator and which will pass through all the steps of the FullSim simulation chain, which are either matched to Jet objects, or may end-up unmatched because of limitations in the matching algorithms and the previous simulation steps. For the moment, we disregard the problem of \emph{fake jets}, that is Jet objects which are not reconstructed from a GenJet object but are instead due to noise or errors in clustering algorithms.

The idea is being able to directly generate correctly distributed Jet objects starting from noise, for stochasticity,  but also from the values of a corresponding GenJet, as a physical-informed input for the network (a process known as \emph{conditioning}): knowing just the diagram-level physics of some process, we are going to skip the Simulation, Digitization and Reconstruction steps.


With the use of \texttt{C/C++} code (name of script) for the \texttt{ROOT} data analysis framework \cite{Brun:491486}, we processed the NanoAOD files and extracted all the Jet objects matched to a GenJet object, across all the events in the file. Because of the large number of variables, we selected a meaningful subset, containing all the necessary information for our test analysis.

First of all, we selected the following 14 GenJet variables for conditioning the generation: 
\graffito{do we want plots of vars?}

\begin{outline}
\1 \emph{The physical properties} of the GenJet, that is \texttt{Eta}, \texttt{Phi}, \texttt{Mass}, \texttt{Pt}, the \texttt{PartonFlavour}, giving the parton content of a GenJet as a specific number and the \texttt{HadronFlavour}, describing the hadron content in a similar way;
\1 \emph{Engineered, physical-informed variables} which we designed to express interesting physical properties of the GenJet. Computing the $\Delta$R separation between the GenJet and the GenMuons present in the event, we selected the first and second \emph{closest muons}, and we computed the following quantities for each one:
\2 \texttt{Dr}, giving the separation from the GenJet, \texttt{DEta}, the $\eta$ difference from the GenJet, \texttt{DPt}, the $p_T$ difference from the GenJet, \texttt{DPhi}, the $\phi$ difference from the GenJet, which is to be computed accounting for the \emph{periodicity} of the $\phi$ variable;

\1 If no GenMuons were present within a cone of $\Delta$R = 0.5 from the GenJet, the corresponding values were set to a user-defined maximum.

\end{outline}

Then, we selected the following 17 target variables for the matched reconstructed Jet objects:

\begin{outline}
\1 \emph{The physical properties} of the Jet \emph{with regard to} the ones of the matched GenJet: \texttt{EtaMinusGen}, the $\eta$ difference , \texttt{PhiMinusGen}, the $\phi$ difference, \texttt{MassRatio}, the ratio of the jet and GenJet masses, \texttt{PtRatio}, the ratio of $p_T$s. This was done because the Simulation and Reconstruction steps are expected to introduce corrections w.r.t. the GenJet distributions, easier to learn when considering these quantities. As an additional variable, the Jet \texttt{Area}, a measure of its susceptibility to radiation, like pileup or underlying event, was added as well;

\graffito{what does btag -1 stand for?}
\1 The most relevant \emph{b-tagging and c-tagging algorithms scores}: \texttt{btagCMVA}, \texttt{btagCSVV2}, \texttt{btagDeepB}, \texttt{btagDeepC}, \texttt{btagDeepFlavB} and \\\texttt{btagDeepFlavC}, which indicate with a score ranging from 0 to 1 whether the Jet contains the respective quark or not, a very significant information for performing event selection during an analysis. Some values may be offsetted to -1 to indicate ???;

\1 The \texttt{bRegCorr}, the $p_T$ correction for b-jet energy regression;

\1 The \texttt{qgl} score for the Quark vs Gluon likelihood discriminator;

\1 The \texttt{jetID} and \texttt{puID} ID flags indicating relevant characteristics of the jet and the Pile-Up.
\end{outline}
\subsection{Muons}

For muons we performed the same procedure, taking only those muons matching to GenMuon objects (a GenParticle object with pdgId value of $\pm$13). 

We selected 30 GenMuon variables for conditioning:

\begin{outline}
\1 \emph{The physical properties} of the GenMuon, that is \texttt{Eta}, \texttt{Phi}, \texttt{Charge} and \texttt{Pt};

\1 \emph{The 14 GenParticle status flags}, a series of \texttt{statusFlags} stored bitwise, with each bit having a different physical interpretation such as \emph{isTauDecayProduct}, \emph{fromHardProcess}, etc. or some information regarding the position of the object in the detector (e.g. \emph{isLastCopy}, indicating that this is the last copy of the GenPart in the detector to be used for the analysis);

\1 \emph{Engineered, physical-informed variables} which we designed to express interesting physical properties of the GenMuon. Computing the $\Delta$R separation between the GenMuon and the GenJets present in the event, we selected the first \emph{closest GenJet}, and we computed the following quantities:
\2 \texttt{Dr}, giving the separation from the GenJet, \texttt{DEta}, the $\eta$ difference from the GenJet, \texttt{Pt}, the $p_T$ of the GenJet,
\texttt{DPhi}, the $\phi$ difference from the GenJet, which is to be computed accounting for the \emph{periodicity} of the $\phi$ variable and finally the \texttt{Mass} of the closest GenJet;

\1 A series of 6 \emph{ Event level variables regarding Pile-Up}:\\ \texttt{Pileup\_gpudensity}, the Generator-level PU vertices/mm,\\ \texttt{Pileup\_nPU}, the number of pileup interactions that have been added to the event in the current bunch crossing, \texttt{Pileup\_nTrueInt}, the true mean number of the poisson distribution for this event from which the number of interactions each bunch crossing has been sampled, \texttt{Pileup\_pudensity}, PU vertices/mm, \texttt{Pileup\_sumEOOT}, the number of early out of time pileup and \texttt{Pileup\_sumLOOT}, the number of late out of time pileup;
\end{outline}

Then we selected 22 target variables for the Muon objects:

\begin{outline}
\1 \emph{The physical properties} of the muon \emph{with regard to} the ones of the matched GenMuon: \texttt{EtaMinusGen}, the $\eta$ difference , \texttt{PhiMinusGen}, the $\phi$ difference, \texttt{PtRatio}, the ratio of $p_T$s. This was done because the Simulation and Reconstruction steps are expected to introduce corrections w.r.t. the GenMuon distributions, easier to learn when considering these quantities. As an additional variable, the \texttt{ptErr}, the $p_T$ error for the muon track, was selected as well;

\1 Six \emph{impact parameters} with regard to the primary vertex: \texttt{dxy}, \texttt{dxyErr}, \texttt{dz}, \texttt{dzErr}, the 3D impact parameter \texttt{ip3d} and its significance \texttt{sip3d}, all expressed in cm;

\1 Some \emph{Boolean flags}: \texttt{isGlobal}, \texttt{isPFcand}, identifying the muon as a Particle Flow candidate, \texttt{isTracker};

\1 A series of \emph{isolation variables} returned by the Particle Flow algorithm: \texttt{pfRelIso03\_all}, \texttt{pfRelIso03\_chg} and \texttt{pfRelIso04\_all};

\1 The \emph{variables related to the closest jet}: \texttt{jetPtRelv2}, indicating the relative momentum of the lepton with respect to the closest jet after subtracting the lepton and \texttt{jetRelIso}, the relative isolation in matched jet;

\1 A series of \emph{ID scores}: \texttt{mediumID}, \texttt{softMVA} score and its cut-based ID \texttt{softMVAId}, \texttt{softId};
\end{outline}

\subsection{Extraction and preprocessing}

With the use of \texttt{C/C++} code (name of script) for the \texttt{ROOT} data analysis framework \cite{Brun:491486}, we processed the NanoAOD files and extracted all the Jet objects matched to a GenJet object and the Muon objects matched to a GenMuon across all the events in the file. This operation can be performed rather quickly thanks to the \emph{compiled} language being used and the powerful \texttt{ROOT::RDataFrame()} class, offering a modern, high-level interface for the manipulation of data stored in a NanoAOD \texttt{TTree}, as well as \emph{multi-threading} and other low-level optimisations.
The output of the \emph{extraction} step is another \texttt{.root} file containing just the selected objects.

The resulting file is still organized according to the Events structure. Besides, we know that many machine learning algorithms work best when specific distributions are \emph{preprocessed} according to specifc criteria. Normalizing Flows are no exception. Specifically, there are four key features which should be accounted for and modified through preprocessing before training:

\begin{figure}
    \centering
    \includegraphics[width=\columnwidth]{gfx/ch5/preproce.pdf}
    \caption[Preprocessing]{Sharply peaked distribution are being converted to more broad ones during the preprocessing step. In this example the \texttt{ip3d} variable gets transformed as log(\texttt{ip3d}+0.001).}
    \label{fig:preproce}
\end{figure}

\begin{outline}
\1 Because NF are learning actual pdfs, \emph{large gaps} between values of the distribution may disturb training and trick the network to \emph{bridge} the extremes of the distribution by creating spurious samples in the gap. When possible, the gaps should be reduced and the values packed closer together;

\1 As NF work with pdfs, they are not well suited to deal with \emph{discrete} distributions. Thus, we should apply a process known as \emph{dequantization}, that is applying some sort of smearing to the discrete values to make them similar to those sampled from a continuous distribution;

\1 For similar resons as before, when possible it would be beneficial to widen and normalize sharply peaked distributions through invertible transforms such as log(x). If well separated, eventual peaks may be dequantized as well;

\1 Finally, we opted for \emph{saturating} long tails of distributions to some limiting values, in order to make it easier for the model to learn the pdf in the more populated region.
\end{outline}

Apart from possibly dequantization, we stress that all of this transformations were implemented to make training easier but are not strictly necessary--the models revealed themselves as powerful enough to deal with complex, sharply peaked, long tailed distributions. However, having already implemented the preprocessing pipeline and because it did not introduce a big overhead in the procedure, we decided to keep it for the present work. An example of one of the possible preprocessing operations is shown in Figure \ref{fig:preproce}.

All of these transforms may be implemented with a clear and natural syntax in the \texttt{Python} programming language, specifically thanks to the \texttt{pandas} package \cite{reback2020pandas}, which implements a convenient dataframe structure. We thus open the \texttt{.root} file directly in a \texttt{Python} script through the \texttt{uproot} package \cite{jim_pivarski_2022_6791281}, discard the Events structure to obtain a simple table with one object per line, perform the preprocessing as discussed above and save the output in \texttt{.hdf5} format for the training.

\section{Models design}
This section describes the implementation  details for the two architectures--the one responsible for the generation of jets and the one targeting muons. We discuss the software choices and then the model specifics and trainings.

\subsection{Software and packages}

We initially planned to use another class of generative models, that is Generative Adversarial Networks. However, after discovering the work of Dr. Stephen Green \cite{stephen_green_2021_4558988} we realized that Normalizing Flows were better suited to our task, as they suffered from less training instabilities and allowed us to directly learn the underlying distributions with an easily interpretable loss function.

The models are implemented following Dr. Green's example: the \texttt{Python} package \texttt{nflows} \cite{conor_durkan_2020_4296287} defines the Classes for Rational Quadratic Spline Normalizing Flows, integrating them for use with the popular ML research package \texttt{Pytorch} \cite{NEURIPS2019_9015}. We obviously had to perform several attempts to optimize the hyperparameters choices for our use case, a long and complicated process which resulted in the architectures presented below.

\subsection{Architectures and trainings}

\begin{figure}
    \centering
    \includegraphics[width=\columnwidth]{gfx/ch5/nfmodel.pdf}
    \caption[Actual NF model]{The NF models are quite large and complex. A number of transforms equal to that of the target variables is performed. The normally distributed inputs \textbf{z} are permuted for each step and then splitted in half, sending half as parameters and half as argument of the \emph{spline transform}. The conditioning variables \textbf{y} are sent as input to a complex 10 layer network (different for each transform) which defines the parameters for the spline. Everithing is repeated until the last step where we permute back to the original order and output the targets \textbf{x}. Where two numbers are separetad by a slash, the first refers to the muons model, the second to the jets one.}
    \label{fig:nfmodel}
\end{figure}

Figure \ref{fig:nfmodel} shows the final models employed in this work. 
As discussed in Section \ref{sec:couplay}, in order to reduce the computational complexity of the Jacobian, we implemented the total transformation as a chain of single, subsequent splines transformations, each acting on just half of the input, while the latter half is kept unchanged and serves as additional parameters for the splines. This has the additional advantage of ensuring good \emph{correlations} between the various variables, as the transformation for one of them will end up depending on every other variable as long as we implement a number of transformation equal to the number of variables and we permute the order linearly before each spline.

For each spline, the normally distributed inputs \textbf{z} are permuted and then splitted in half, sending half as parameters and half as argument of the \emph{spline transform}. The conditioning Gen-level variables \textbf{y} are sent as input to a complex 10 layer fully-connected network (a different one for each transform) which defines the parameters for the spline. Its most relevant hyperparameters are the \emph{hidden\_dim}, the number of nodes per hidden layer, set to 256 for the muons model and to 298 for the jets one, the \emph{n\_bin}, the number of bins for the spline, set to 128 for both models and the \emph{n\_blocks} set to 10 for both and defining the number of hidden layers.
Each network was defined with \texttt{ReLU} activation function and setting \texttt{batch\_norm=True}.
\graffito{need to discuss batchnorm AND cosine\_annealing in ch3}

We would like to emphasize the fact that because \emph{each transform} defines a separate network for learning the optimal spline parameters, the final models are quite large, especially for physics-based application standards. The muons model exceeds 54e6 trainable parameters (54,988,164 parameters, corresponding to the various weights of the neurons in each layer), while the jets model totals in at 47,986,595 parameters. These large models were trained on around 5e6 muons or jets objects extracted as explained above, and the \emph{losses} were monitored at each epoch on a separate \emph{validation set} of about 4e5 samples to ensure that the models were not being over-optimized for the training set.

As our \emph{optimizer} algorithm we choose the de-facto standard in the field, the \texttt{Adam} algorithm \cite{https://doi.org/10.48550/arxiv.1412.6980}, a complex and powerful algorithm for models optimization still based on the same basic principles of Section \ref{sec:backprop}, with an initial \emph{learning rate} of xxx, reduced during training by the \emph{cosine annealing} procedure.


\begin{figure}
    \myfloatalign
    \subfloat[]
    {\includegraphics[width=.45\linewidth]{gfx/ch5/lossesmuons.png}} \quad
    \subfloat[]
    { \includegraphics[width=.45\linewidth]{gfx/ch5/lossesjets.png}} \\
    \caption[Models losses]{stuff.}\label{fig:losses}
    
\end{figure}

\section{Results}

\subsection{1d distributions and correlations}

\subsection{Conditioning}

\section{A prototype end-to-end analysis sample generator}

%************************************************
\chapter{Conclusion and future outlooks}\label{ch:outlook} % $\mathbb{ZNR}$
%************************************************
%************************************************
\chapter{Conclusion and future outlook}\label{ch:outlook} % $\mathbb{ZNR}$
%************************************************
In the present work, we introduced and implemented an alternative paradigm for the flash generation of events at the CMS experiment, based on state-of-the-art results from the ML field.

We were able to successfully simulate physical target distributions with great accuracy, preserving the correct correlations between them and demonstrating that our method is capable of varying the output results according to the physical information provided as input. This conditioning has been compared to that of the other competing approach for fast simulation, CMS FastSim, and has been shown to provide more accurate and precise results when compared with the original FullSim target. Additionally, the proposed solution has demonstrated decisive advantages in terms of speed, with orders of magnitude of speed-up thanks to modern, GPU-accelerated computing.

We also built a first prototype for an end-to-end sample generator in the standard CMS NanoAOD format, and actually deployed it at scale on millions of events, coming from the training physical process as well as new, previously unseen ones. 

Finally, we demonstrated that the results obtained can actually be used in a real world scenario such as a complex, multivariate, MC based analysis as the VBF Channel of H$\rightarrow\mu^+\mu^-$. We computed the key derived quantities for the analysis for both the FullSim and the FlashSim samples, and we observed comparable results on the evalutation of the DNN classifier actually used in the corresponding CMS publication.
Our approach was also able of providing interesting results regarding the calculation of higher order QCD effects, being capable of reproducing the differences between competing approaches such as \texttt{POWHEG} and \texttt{MadGraph5\_aMC@NLO}.

\section{Towards FlashSim}
This work has also emphasized some limitations and peculiarities of the selected approach, pointing to the next steps to be take if the CMS Collaboration were to adopt the proposed method.

The two major ones are discussed below.


\subsection{Building a full NanoAOD}
A full scale FlashSim must be able of reproducing the content of a NanoAOD file in its entirety. Due to the hundreds of variables stored in a single Event, the most reasonable approach is the one taken in the present work: instead of devising a massive single model for generating all the variables, it is better to divide the problem into a collection of separate physical objects, each reconstructed through its own network.

This has the clear advantage of reducing the size of the models and the resources needed for their trainings, beside, each single model may be conditioned on relevant quantities coming either from the Gen-level, from other model outputs or being specifically engineered to communicate key information regarding the event.

One crucial element for the production of convincing NanoAODs is the presence of \emph{fakes}. The stochastic nature of these objects make their production non-trivial, however there exist other techniques from the ML field, such as \emph{LSTM} \cite{lstm}, for the handling and the production of variable-length sequences. The generation would obviously depend on key quantities such as the PileUp and the GenParticles.

Another key quantity, the \emph{Missing Transverse Energy} (MET), could also depend on the final-state reconstructed objects, as this would allow us to obtain a more consistent description of the whole event.

A possible global picture, with multiple dependencies and conditionings, is illustrated in Figure ???

\subsection{Optimization}
An important series of steps in the prototype end-to-end generator need to be optimized. 
First of all, the Gen-level quantities currently extracted from existing FullSim NanoAODs and written to file for processing could actually be produced from a dedicated FlashNano-Generator, capable of passing the desired inputs directly to our models without the need of previously existing files.

The ML models forming the backbone of our approach need to be properly optimized and thoroughly examined to find the best combination of hyperparameters, size and performance. Additionally, the training and the generation procedure may be modified to be run in parallel over multiple machines or clusters, providing significant speed adavantages. Another very serious issue to be addressed by the collaboration is the fact that our research is based on a series of \texttt{Python} packages maintained by a series of open-source contributors with no interest in the specific HEP use case and no assurance of continued and prolonged support, as well as backward compatibility.

The issue may be addressed in several ways. A possibility would be to experiment with splines-based models such as ours with the current packages, but once found the optimal transformation, the splines and their parameters could be mapped into a more convenient \texttt{C/C++} framework to be used and actively maintained by one of the computing groups of CERN, and possibly integrated as part of the \texttt{ROOT} language.

Additionally, the postoprocessing step needs to be revised with the addition of a check for numerical instabilities causing nonphysical values for the target distributions. These may be then easily corrected by regenerating the faulty event with another set of random noise as input.

\section{Future research}
The need for fast and reliable computing methods in the physical sciences, notably in the high-energy field, has fueled much of the technological progress of modern times. Despite this, the average physicist has generally little time to spend on innovating and improving its computing toolkit, and has to content himself with boilerplate solutions. This is especially true in highly specialized fields, such as HEP, which purse a wide variety of research directions.

Specifically, in recent years, \emph{Machine Learning} techniques have been massively adopted by scientific collaborations around the world. However, such tools remain geared towards the necessities of industry; much work remains to be done to enable the use of this technologies in hard sciences.

As physicists with a keen interest in this type of applications, while still retaining useful domain knowledge, we are convinced that significant results could be achieved by pursuing the following research directions. 

\subsection{Other Flow-based applications in HEP}

The versatility of Normalizing Flows make them an optimal choice for tackling a wide range of problems, namely all those where the definition and manipulation of \emph{pdf}s is of vital importance.

We already mentioned a series of possible approaches in Section \ref{sec:nfapp}. The most interesting is possibly the approach to \emph{anomaly detection}, where Flows have already been used to define empirical distributions from data (see \cite{Kasieczka_2021}). Other interesting approaches being currently tested and deployed at CERN make use of ML approaches as unbiased function approximants for unknown, empirical pdfs (e.g. \cite{D_Agnolo_2019}), however, perhaps being generally less known, NF have yet to be tested as a solution to the problem.

\subsection{Graph Normalizing Flows}

Possibly considered the holy grail of ML for HEP applications at the LHC, the problem of \emph{track reconstruction} is mostly a pattern recognition task whose complexity grows much more than linearly with the increase of number of collisions, hence the number of tracks, and is thus expected to be one of the main problems for HL-LHC. When investigating such a large feature space, a common approach is to reduce the complexity of the network while still retaining useful information about local features through the use of \emph{Convolutional Neural Networks} (CNNs), which nowadays are the standard approach for image datasets (as pioneered in \cite{simonyan2015deep}).
Unfortunately, any representation of tracking detector as images
would look very sparse and would not benefit of the locality idea of the CNN. 

Another hard HEP problem is the reconstruction of secondary vertices used for b-tagging. In order to identify b-jets, a useful feature is the presence of the so called \emph{secondary vertices} in a jet, i.e. points in space, displaced from the interaction point, where a group of tracks
appears to originate from.

A promising way to overcome these challenges is the introduction of \emph{Graph Networks} (GNs), which represent input and output data as graphs to exploit any invariance of the graph itself in order to perform the dimensionality reduction that is achieved in CNN. In the case of a GN, the locality is not based on an euclidean metric like as before, but rather on the number of connections needed to reach one node from another. In order to enforce this kind of locality, a \emph{Message Passing} schema is often used for GN: the computation happens in several iterations where each iteration propagates information to/from neighbor nodes. Even a simple overview of the method goes beyond the scope of this section: we will now comment briefly on specific HEP advantages of this architecture, referring the reader to the seminal work of Battaglia \emph{et al.} \cite{battaglia2018relational} for a comprehensive review.

The key idea behind GNs in HEP is to represent the data as a graph where the nodes are the hits (i.e. the individual measurements from tracking sensors) and nodes of subsequent layers are connected (with some pruning of nonphysical connections). The output is instead a graph made of several disconnected branches each representing an
individual track (or track seed). We would like to emphasize how a \emph{graph input topology} would possibly benefit many other models already in use at LHC, allowing a better application simply by performing a preprocessing step through the use of GN layers. An interesting work in this direction, proposing a new approach that considers a jet as an unordered set of its constituent particles, effectively a "particle cloud" is the one presented by Qu and Gouskos \cite{pj2020}. Based on the particle cloud representation, they proposed ParticleNet, a customized neural network architecture using Dynamic Graph Convolutional Neural Network for jet tagging problems. The ParticleNet architecture achieves state-of-the-art performance on two representative jet tagging benchmarks and is improved significantly over existing methods. 

The graph topology may be extended to the Normalizing Flow approach as well. The autors of \cite{https://doi.org/10.48550/arxiv.2105.09016} introduced \emph{Equivariant Normalizing Flows} based on graph networks as the building block for defining equivariant invertible functions acting on graphs and capable of translating the NF approach to the generation of molecular structures.

In a similar vein, we could imagine to generate graph structures representing our target particle cloud representation, paving the way for an entirely new and powerful simulation approach at a completely different level from what discussed in the present work.

\subsection{Quantum Machine Learning}

Finally, a novel discipline born from \emph{Quantum Computing} and ML, known as \emph{Quantum Machine Learning} (QML) is already under serious investigation from the scientific community, due to the potential and significant \emph{quantum} advantages. The limits of what machines can learn have always been defined by the computer hardware we run our algorithms on—for example, the success of modern-day deep learning with neural networks is enabled by parallel GPU clusters.

Quantum machine learning extends the pool of hardware for machine learning by an entirely new type of computing device—the \emph{quantum computer}. Some research focuses on ideal, universal quantum computers (“fault-tolerant QPUs”) which are still years away. But there is rapidly-growing interest in quantum machine learning on near-term quantum devices (\emph{Noisy Intermediate-Scale Quantum} or NISQ).

We can understand these devices as special-purpose hardware like Application-Specific Integrated Circuits (ASICs) and Field-Programmable Gate Arrays (FPGAs), which are more limited in their functionality but nonetheless well suited to specific applications. In the modern viewpoint, quantum computers can be used and trained like neural networks. We can systematically adapt the physical control parameters, such as an electromagnetic field strength or a laser pulse frequency, to solve a problem. Additionally, quantum circuits are differentiable, and a quantum computer itself can compute the change in control parameters needed to become better at a given task. Trainable quantum circuits can be leveraged in other fields like quantum chemistry or quantum optimization. It can help in a variety of applications such as the design of quantum algorithms, the discovery of quantum error correction schemes, and the understanding of physical systems.

At the moment the effort of the HEP community is focused on quantum generative models (see \cite{chan2021quantum}), where the noisy behavior is mitigated or even beneficial. The CERN has already established a partnership with IBM, a leading competitor for quantum technologies, and founded the CERN Quantum Technology Initiative (CERN QTI), a comprehensive R$\&$D, academic and knowledge-sharing initiative to exploit quantum advantage for high-energy physics and beyond. 
%\include{multiToC} % <--- just debug stuff, ignore for your documents
% ********************************************************************
% Backmatter
%*******************************************************
\appendix
%\renewcommand{\thechapter}{\alph{chapter}}
\cleardoublepage
\ctparttext{You can put some informational part preamble text here.
Illo principalmente su nos. Non message \emph{occidental} angloromanic
da. Debitas effortio simplificate sia se, auxiliar summarios da que,
se avantiate publicationes via. Pan in terra summarios, capital
interlingua se que. Al via multo esser specimen, campo responder que
da. Le usate medical addresses pro, europa origine sanctificate nos se.}
\part{Appendix}
%********************************************************************
% Appendix
%*******************************************************
% If problems with the headers: get headings in appendix etc. right
%\markboth{\spacedlowsmallcaps{Appendix}}{\spacedlowsmallcaps{Appendix}}
\chapter{Appendix Test}\label{ch:appx}
First of all, we point out to the interested reader that the final version of the code used in this Thesis is hosted and documented in detail online \href{tbd}{at the following repository}\footnote{repository link, tbd}.

%Errem omnium ea per, pro congue populo ornatus cu, ex qui dicant
%nemore melius. No pri diam iriure euismod. Graecis eleifend
%appellantur quo id. Id corpora inimicus nam, facer nonummy ne pro,
%kasd repudiandae ei mei. Mea menandri mediocrem dissentiet cu, ex
%nominati imperdiet nec, sea odio duis vocent ei. Tempor everti
%appareat cu ius, ridens audiam an qui, aliquid admodum conceptam ne
%qui. Vis ea melius nostrum, mel alienum euripidis eu.

\section{Appendix Section Test}
Test: \autoref{tab:moreexample} (This reference should have a
lowercase, small caps \spacedlowsmallcaps{A} if the option
\texttt{floatperchapter} is activated, just as in the table itself
 $\rightarrow$ however, this does not work at the moment.)

\begin{table}[h]
    \myfloatalign
    \begin{tabularx}{\textwidth}{Xll} \toprule
        \tableheadline{labitur bonorum pri no} & \tableheadline{que vista}
        & \tableheadline{human} \\ \midrule
        fastidii ea ius & germano &  demonstratea \\
        suscipit instructior & titulo & personas \\
        %postulant quo & westeuropee & sanctificatec \\
        \midrule
        quaestio philosophia & facto & demonstrated \\
        %autem vulputate ex & parola & romanic \\
        %usu mucius iisque & studio & sanctificatef \\
        \bottomrule
    \end{tabularx}
    \caption[Autem usu id]{Autem usu id.}
    \label{tab:moreexample}
\end{table}

%Nulla fastidii ea ius, exerci suscipit instructior te nam, in ullum
%postulant quo. Congue quaestio philosophia his at, sea odio autem
%vulputate ex. Cu usu mucius iisque voluptua. Sit maiorum propriae at,
%ea cum primis intellegat. Hinc cotidieque reprehendunt eu nec. Autem
%timeam deleniti usu id, in nec nibh altera.




\section{Another Appendix Section Test}
Equidem detraxit cu nam, vix eu delenit periculis. Eos ut vero
constituto, no vidit propriae complectitur sea. Diceret nonummy in
has, no qui eligendi recteque consetetur. Mel eu dictas suscipiantur,
et sed placerat oporteat. At ipsum electram mei, ad aeque atomorum
mea. There is also a useless Pascal listing below: \autoref{lst:useless}.

\begin{lstlisting}[float=b,language=Pascal,frame=tb,caption={A floating example (\texttt{listings} manual)},label=lst:useless]
for i:=maxint downto 0 do
begin
{ do nothing }
end;
\end{lstlisting}

%Ei solet nemore consectetuer nam. Ad eam porro impetus, te choro omnes
%evertitur mel. Molestie conclusionemque vel at, no qui omittam
%expetenda efficiendi. Eu quo nobis offendit, verterem scriptorem ne
%vix.


%********************************************************************
% Other Stuff in the Back
%*******************************************************
\cleardoublepage\include{FrontBackmatter/Bibliography}
\cleardoublepage\include{FrontBackmatter/Declaration}
\cleardoublepage\include{FrontBackmatter/Colophon}
% ********************************************************************
% Game Over: Restore, Restart, or Quit?
%*******************************************************
\end{document}
% ********************************************************************
